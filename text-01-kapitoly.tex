% ################################
\section{Úvod}

% todo podle tohoto upravit ještě cíl projektu a abstrakci
% todo cílem je taky udělat high level pohled na moderní vývoj web aplikací a jejich možných technologií a způsobů vývoje

Cílem této bakalářské práce je navrhnout a hlavně implementovat webovou aplikaci fungující na všech standardních zařízení
pro vzájemné sdílení a vyhledávání kontaktních a doplňujících údajů, sociálních sítí a firemní struktury mezi osobami, firmami,
neziskovými organizacemi, umělci a jinými subjekty na jednom místě.
Hlavním požadavkem je umožnit snadný a rychlý přístup ke všem informacím všem bez omezení jako je registrace nebo
geografická lokace jedince.
I přesto, že existuje spousta podobných webových aplikací, žádná nekombinuje všechny vlastnosti a funkce navrhované
v této práci, nebo se zaměřuje na jinou cílovou skupinu.
Cílem není pouze navrhnout a implementovat webovou aplikaci obsahující všechny navrhované funkce, velmi důležitým
požadavkem je propracované uživatelské prostředí s důrazem na design a uživatelskou přívětivost při zachování kompatibility
a přívětivosti aplikace na mobilních zařízeních i počítačích.

% todo je to pouze rozcestník ne sociální síť se zprávami, feedem, statusy a podobně
% digitální vizitky

Velmi důležitým krokem před samotnou implementací je zvolení vhodných technologií umožňující nejsnazší a nejrychlejší
vývoj a hlavně implementaci všech požadavků.

% ################################
\section{Konkurenční a podobná řešení}

Jak již bylo zmíněno, existuje spoustu webových aplikací, které by se dali považovat za konkurenční řešení implementované
v této práci.
Nicméně žádná momentálně nekombinuje prvky sdílení kontaktních údajů, sociálních sítí a firemní struktury s plně
fulltextovým vyhledáváním bez nutnosti registrace nebo rušení sociálními prvky jako jsou zprávy nebo feedy novinek
náhodných osob a firem, případně ukládání subjektů do oblíbených s možností kategorizace či poznámek nebo globálnosti
platformy.
U většiny takových aplikací se totiž spoléhá na navigaci na cílené subjekty pomocí přímých odkazů, které si daný subjekt
vloží na své sociální sítě nebo vizitky.

% TODO
	\subsection{Linktree}
	- neumožňuje vyhledávat
	- neumožňuje ukládat do oblíbených
	- nemá hezké UI
	- často pochybné odkazy
	- spousta integrací do ostatních aplikací
	- pokročilé statistiky
	- produkty

	\subsection{LinkedIn}

	- vyžaduje registraci
	- působí spíš jako samostatná sociální síť, ne jako rozcestník
	- spam od cizích uživatelů a firem
	- feed s reklamami
	- složitě dohledatelná firemní struktur
	- cílí na firmy a zaměstnance ne na osoby, umělce nebo eventy apod

	\subsection{Firmy.cz}

	- není globální
	- soustředí se pouze na firmy

	\subsection{Shorby}

	- neumožňuje vyhledávat
	- neumožňuje ukládat do oblíbených

	\subsection{Zlaté stránky}

	TODO

	\subsection{Swopi}

	TODO

	\subsection{AllMyLinks}

	TODO

% ################################

\section{Návrh vlastního řešení}

Jak již bylo nastíněno, navrhované řešení bude reprezentovat jakýsi rejstřík pro sdílení a vyhledávání základních informací
subjektů a objektů, které uživatele dovedou k hledaným informacím.
Základní obecnou jednotkou reprezentující takový subjekt či objekt bude karta; odvozenina z anglického překladu vizitky "business card".
Každá karta tak bude jakousi digitální pokročilou náhradou klasických vizitek nevztahující se pouze na osoby a jejich
hlavní kontaktní údaje.
Cílem je umožňovat vyhledávat a sdílet informace kromě soukromých osob a firem i události, produkty, místa, umělecká díla a mnohem více.
Jedinou podmínkou pro takový subjekt nebo objekt je existence alespoň jednoho údaje důležitého pro ostatní osoby.
Tím může být kontaktní údaj, odkaz na jiné webové stránky (např.: sociální sítě), geografická pozice nebo
nějaká hierarchie ostatních subjektů či objektů.

	\subsection{Oblíbené karty}

	Ostatní uživatelé pak budou mít možnost takové karty, které je nějakým způsobem zajímají, uložit do svého profilu pro
	zjednodušení zpětného dohledání například kontaktních údajů.
	Kromě obyčejného uchování oblíbených karet bude možné takové karty seskupovat do vlastních upravitelných složek pro
	lepší organizaci a komentovat vlastními soukromými poznámkami.
	Každá oblíbená karta tak bude příslušet nějaké složce a bude mít poznámku s pokročilým formátováním viditelnou pouze
	autorovi poznámky.
	Tyto funkce umožní uživatelům komunikujícím s velkým počtem subjektů vyznat se v kontaktních údajích a poznámkách
	například s poslední komunikace nebo podrobností o daném subjektu bez nutnosti ručně takové informace spravovat v
	textových editorech nebo papírových poznámkách.

	\subsection{Viditelnost karet}

	Protože vyplnění takové karty může být zdlouhavý proces, i za předpokladu intuitivního \ac{UI}, je potřeba umožnit
	uživatelům karty skrýt do té doby než budou připravené ke zveřejnění.
	Zároveň ale takovou rozpracovanou kartu může uživatel chtít někomu vzdáleně ukázat například pro ověření správnosti informací
	bez zveřejňování.
	Karty tak budou moct mít dvě viditelnosti: veřejná a skrytá.
	Veřejná karta bude mít jednoduchou zapamatovatelnou URL adresu a bude ji možné vyhledávat pomocí fulltextového vyhledávače.
	Skrytá karta nebude vyhledatelná pomocí vyhledávače, nicméně uživatelé vlastnící URL odkaz na danou kartu budou mít přístup
	ke zobrazení dané karty.
	Skrytá karta bude mít navíc možnost vygenerování náhodné URL pro znemožnění uhádnutí URL adresy nepovolenou osobou.

	\subsection{Karetní informace}

	Každá karta, jak již bylo nastíněno, bude mít možnost zobrazovat různé informace pro uživatele.
	Definovatelné informace budou vždy dané, tj. budou mít vlastní strukturu v kódu, validaci správnosti hodnot a specializované
	\ac{UI}.
	Nebude možné uživatelem definovat náhodné informace, aby se předešlo ke zhoršení čitelnosti informací karet.
	Toto rozhodnutí by mělo pomoci především uživatelům nepříliš zdatných v dodržení přehlednosti většího množství
	informací za pomoci předem daného formátu zobrazení daných informací.
	Je však potřeba strukturu navrhnout dostatečně univerzálně, aby bylo možné v budoucnu implementovat další typy informací
	definovatelných na kartách.
	Momentálně podporovanými takovými informacemi budou:
	\begin{itemize}
		\item fotografie/avatar
		\item kategorizační štítky
		\item popisek
		\item základní kontaktní údaje
		\item odkazy na profily na sociálních sítí
		\item generické odkazy na jakékoliv externí webové stránky
		\item geografická lokace
		\item hlavní otevírací doba
		\item kanceláře, pobočky
	\end

	Pro rychlé vizuální rozpoznaní karty bude mít autor k dispozici možnost nahrát vlastní fotografii nebo avatar s logem.
	Fotografie nebo avatar dokáže dostatečně odlišit jednotlivé karty bez nutnosti čtení názvu nebo popisku ke zjistění o
	jakou kartu se vlastně jedná.

	Kategorizační štítky budou sloužit pro globální kategorizaci karet pro úvodní povědomí uživatele o jakou kartu se jedná
	a hlavně pro vyhledání karet ve stejné kategorii.
	To se může hodit pro seskupení například osob s určitým povolání nebo vyznávaným životním stylem.
	Konkrétní využití je však na konkrétních uživatelích.
	Štítku uživatel bude moct definovat několik nebo žádný.

	Popisek bude sloužit především pro stručnou definici obsahu karty, tj. co daná karta reprezentuje.
	Předpokládá se tedy, že popisek karty bude zodpovídat některé z následujících otázek?
	\begin{itemize}
		\item Je to osoba, firma, událost, dílo nebo předmět?
		\item Čím daný subjekt či objekt zabývá v profesní nebo soukromé sféře?
		\item Proč je daný subjekt či objekt zajímavý?
	\end
	Popisek bude mít omezenou velikost a nebude povinný, nicméně silně doporučovaný.

	Základními kontaktními údaji se v případě těchto karet rozumí: telefonní číslo, hlavní emailová adresa, hlavní
	webové stránky a identifikační číslo (\noindent\ac{IČO}).
	Uživatel si pak vybere jaké z těchto údajů bude chtít zveřejnit a jaké nikoliv.

	Pro většiny karet, zejména pak pro karty reprezentující soukromé osoby, budou nejdůležitější odkazy na sociální sítě.
	Pro jednoduché rozlišení mezi jednotlivými sítěmi a poskytnutí koncovým uživatelům jistotu, že daný odkaz skutečně
	směřuje na ověřenou doménu dané sociální sítě, systém bude mít předefinované podporované sociální sítě.
	Každá taková definice bude jakousi šablonou pro výsledné odkazy na sociální sítě a bude obsahovat oficiální název,
	názvy podle kterým bude možné sít vyhledat, ikonu, šablony validních URL adres a kategorii.
	Uživatel pak bude moct při tvorbě karty vybrat konkrétní síť ručně ze seznamu nebo nechat aplikaci nalézt správnou
	síť podle vložené URL adresy.
	Šablona sítí budou kategorizované a bude možné je vyhledávat podle názvu pro jednodušší dohledání.
	Po vložení konkrétní URL adresy vznikne konkrétní odkaz držící informaci o použité šabloně, cílové URL adresy a
	popisku pro rozeznání různých odkazů na stejné sociální sítě.
	Díky definovaným šablonám reprezentující rozsah validních URL pro konkrétní sociální síť bude aplikace schopna
	validovat konkrétní URL adresy zda-li odpovídají alespoň jedné šabloně a jestli zadaná URL adresa skutečně existuje.
	To jednak umožní autory karet upozornit na nesprávné URL adresy a koncovým uživatelům karet poskytne zmiňovanou jistotu
	URL adresy skutečně vedoucí na prezentovanou síť.

	Generické odkazy budou podobné odkazům na sociální sítě.
	Rozdíl však bude ve volnosti zadat jakoukoliv validní URL adresu.
	Taková adresa sice bude stále validovaná aplikaci pro existenci, nebude už však validovaná proti žádnému rozsahu
	URL adres.
	To umožní autorovi karty odkazovat na jakékoliv webové stránky, ale znemožní poskytnutí jistoty pravosti odkazu
	pro koncové uživatele.
	Bohužel není možné jednoduše strojově zjistit jestli odkaz nevede na podvodnou stránku.
	Takové stránky jsou obtížné rozeznat od originálních webových stránek i pro pravidelné uživatelé originálních stránek.
	Navíc podvodné stránky každým dnem přibývají a autoři takových webů jsou stále sofistikovanější.

	Důležitou informací pro firmy, události, pobočky atd. je geografiká lokace sídla.
	Ta umožní koncovým uživatelům nalézt subjekt či objekt, reprezentovaný kartou, v reálném světě.
	Zároveň poskytne možnost zobrazit karty v blízkém okolí daného uživatele pomocí mapy.
	Uživatel tak jednoduše zjistí jaké firmy, osoby nebo události se nachází v okolí jeho bydliště, zaměstnaní nebo
	cílové destinace.

	Firmy a obchody většinou mají nějakou provozní/otevírací dobu, kdy jsou zaměstnanci schopni obsloužit své zákazníky.
	Tato informace je navíc uživateli vyhledávaná opakovaně a je tedy nutné, aby byla po ruce a zobrazovala všechny potřebné
	informace.
	Každá karta tak bude moct definovat otevírací dobu každého dne v týdnu s volitelnou přestávkou.
	Uživatelé tak kromě definovaných otevíracích dob uvidí i informaci o tom, jaká otevírací doba pro daný den platí
	a za jak dlouho daný subjekt bude mít otevřeno, popřípadě kolik času zbývá do konce otevírací doby.

	Nedílnou součástí větších firem a institucí jsou pobočky a kanceláře rozmístěné v různých koutech světa.
	Každá taková pobočka má své zaměstnance, geografickou lokaci a v některých případech i vlastní otevírací dobu.
	Autor karty tak může definovat své pobočky či kanceláře a volitelně je obohatit o zmíněné rozšiřující informace.
	Každá taková pobočka může mít vlastní otevírací dobu, i přes to, že samotná karta může mít již hlavní otevírací dobu
	definovanou, nebo nemusí mít žádnou.
	Pro klienty je pak vhodné mít definovanou i lokaci dané pobočky.
	Podstatnou částí bude možnost přiřadit zaměstnance nacházející se v dané pobočce pro snadné kontaktování přímo konkrétních
	osob.
	Zaměstnanec bude mít definovanou pracovní pozici a jméno, volitelně pak ještě fotografii a základní kontaktní údaje
	v podobě emailové adresy telefonního čísla.
	Každý takový zaměstnanec bude reprezentován speciálním typem karty pro možnost uložení konkrétních zaměstnanců mezi oblíbené
	karty.

	\subsection{Typy karet}

	Aby bylo možné definovat pro karty různého zaměření jiné podporované informace, bude existovat podpora pro typy karet.
	Každý typ bude definovat účel karty, možné informace a platné použití.
	Koncoví uživatelé budou schopni díky této univerzalitě ukládat karty mezi oblíbené bez rozdílů.
	Momentálně dostupnými typy budou: obecná karta a karta zaměstnance.
	Obecná karta bude umožňovat specifikovat všechny výše zmíněné informace a každý uživatel ji bude moct přímo vytvořit.
	Karta zaměstnance bude automaticky nepřímo tvořena při tvorbě zaměstnanců poboček.
	To umožní pro zobrazit detail jednotlivých zaměstnanců, mít unikátní odkaz na každého zaměstnance nebo vyhledávat
	přímo karty zaměstnanců s referencí na rodičovskou firmu.
	Typy karet se ale mohou v budoucnu rozrůst o další specializované typy, a je proto nutné s takovým rozšířením při
	implementaci počítat.

	\subsection{Vyhledávání karet}

	Velmi podstatnou součástí výsledné aplikace je schopnost vyhledávat a objevovat karty podle požadavků koncového uživatele.
	Pro zajištění těchto funkcí bude aplikace umět fulltextově vyhledávat a geograficky zobrazovat karty na mapě světa.
	Díky fulltextovému vyhledávání bude uživatel schopen vyhledávat chtěné karty podle frází obsažených v názvu,
	popisku, kontaktních údajích, odkazech, názvů poboček a karet zaměstnanců.
	Výsledky takového vyhledávání budou seřazený podle relevantnosti vzhledem k hledané frázi.
	Mimo konkrétního vyhledávání uživateli bude zobrazena mapa světa se zaměřením na jeho současnou lokaci zobrazující
	karty a pobočky jako body v mapě.
	Každý bod bude obsahovat souhrn nejdůležitějších informací z reprezentující karty nebo pobočky.
	Takovou informací může být název, štítky, otevírací doba konkrétního dne, odkazy nebo zaměstnanci.
	Díky těmto funkcím budou uživatelé schopni objevovat nové subjekty a objekty bez nutnosti znalosti URL adres
	konkrétních karet.

	\subsection{Dostupnost aplikace}

	Cílem aplikace je poskytnout globálně komukoliv bez omezení přístup k veřejným informacím.
	Na rozdíl od některé konkurence nebude vyžadováno po uživatelích registrace ve výsledné aplikaci pro přístup k vyhledávání
	nebo samotným kartám a jejich informacím.
	Jedinou opodstatněnou výjimkou pro nutnost registrace by se v budoucnu mohli stát karty s příznakem 18+.
	Registrovaný účet uživatele by pak sloužil pro ověření věku daného uživatele podle data narození.
	Tento požadavek však současné verzi navrhované aplikace není řešen.
	Dalším požadavkem je umožnit přístup k informacím z jakékoliv země světa.
	Budoucím cílem je postupně lokalizovat aplikaci do dalších jazyků pro ještě větší dostupnost v zemích mluvících
	jiným než anglickým jazykem.
	Nicméně chybějící jazyk dané země nesmí být sám o sobě důvod pro nedostupnost aplikace v dané zemi, uživatelé by
	měli být vždy schopni používat minimálně výchozí anglickou lokalizaci.

	\subsection{Uživatelské účty}

	Jak již bylo v předchozí kapitole nastíněno, uživatelské účty budou volitelnou funkcí aplikace přinášející výhody
	nemožné implementovat bez existující účtu.
	Mezi takové výhody bude z počátku patřit tvoření karet a ukládání cizích karet mezi oblíbené.
	Tyto funkce vyžadují uživatelský účet, aby bylo možné určit kdo má právo upravovat a mazat dané vytvořené karty
	a ukládat soukromé data uživatele jako jsou oblíbené karty a jejich poznámky bez viditelnosti ostatním uživatelům.
	Uživatelské účtu jsou tak tedy spíše bezpečnostním nástrojem pro správu dat v tvořené webové aplikaci, nikoliv
	nástrojem pro získání dat uživatelů pro například pro zpracování nesouvisející se záměrem aplikace nebo prodej dat
	společnostem zabývajících se inzercí reklamy.
	Uživatelské účty jako takové nebudou reprezentovat vždy jen jednu kartu.
	Místo toho každý účet bude moct vytvořit několik karet a účet bude představovat jen vlastníka neviditelného pro
	veřejnost.
	To umožní uživatelům s potřebou více karet pro různé subjekty či objekty používat jeden registrovaný účet místo
	separátního pro každou kartu jako tomu bývá zvykem u konkurence.

	Kromě těchto hlavních funkcí uživatelských účtů, správná aplikace musí svým zaregistrovaným uživatelům poskytnout
	i nástroje pro správu účtů jako takových.
	Každý uživatelský účet bude identifikován jednoznačně pouze podle emailové adresy.
	Aplikace nebude vyžadovat žádné další údaje v podobě jména, bydliště a podobně, protože pro tyto nejsou potřeba pro
	provoz aplikace a aplikace nebude zpracovávat uživatelská data jiným způsobem.
	Emailová adresa místo uživatelského jména byla zvolena z důvodu možnosti informovat uživatele o různých událostech
	v aplikaci bez nutnosti otevírat samotnou webovou aplikaci.
	Takové události mohou zahrnovat potvrzení změny emailové adresy, upozornění na změnu hesla nebo třeba vyžádání obnovy
	zapomenutého hesla.
	Mimo samotné registrace a přihlášení, aplikace bude umožňovat změnu emailu bezpečně obnovit zapomenuté heslo, kompletně
	smazat vytvořený účet se všemi jeho daty nebo odhlásit uživatele ze všech přihlášených zařízení.
	Přihlášení pomocí emailové adresy a hesla bude navíc rozšířeno o přihlášení a registraci skrze účty třetích stran.
	To zjednoduší a hlavně zrychlí registraci a přihlášení uživatelů využívající služby Googlu, Facebooku nebo třeba
	GitHubu, kde již daní uživatelé můžou být zaregistrovaní.
	Při registraci v aplikaci pomocí některého z poskytovatelů pak poskytovatel předá aplikaci základní údaje automaticky
	a uživatel pouze registraci potvrdí bez nutnosti cokoliv ručně vyplňovat.

		\subsubsection{Prémiové funkce}

		Aby bylo možné obecně jakoukoliv webovou aplikaci dlouhodobě provozovat, je nutné zajistit financování provozu
		produkčního prostředí.
		Možností je velké množství a záleží na představivosti vlastníka aplikace.
		Nejčastějšími praktikami jsou: prodej produktů nebo služeb, zobrazování reklamy a placené plány s prémiovými funkcemi.
		V případě této aplikace by bylo možné nabízet a prodávat fyzické produkty v podobě například speciálních čipových karet
		konfigurovatelných pro přesměrování na uživatelem definovaných karet.
		Avšak prodej jakýchkoliv fyzických produktů není hlavním cílem této aplikace, zejména kvůli přidané zátěži
		pro provoz.
		Zobrazování reklam při navigování aplikací by bylo možné a nepřidalo by toto řešení nadbytečnou zátěž na provoz,
		nicméně uživatelské rozhraní by značně trpělo svojí razantně sníženou přehledností a nechtěným proklikům zobrazovaných
		reklam což by mohlo vést ještě k frustraci uživatelů.
		Existuje navíc značné procento uživatelů využívající speciální nástroje prohlížečů pro blokování všech reklam.
		Ideální variantou pro tento typ aplikace se jeví zavedení placených plánu s prémiovými funkcemi.
		Zaregistrovaný uživatel standardně bude mít bezplatný plán poskytující uživateli základní omezené funkce, ale
		bude moct si aktivovat prémiový placený plán odemykající všechny zbylé funkce co aplikace bude nabízet.
		Toto řešení zamezí zhoršení uživatelského rozhraní a celkově nebude omezovat běžné koncové uživatele.
		Zároveň řešení nebude ovlivněno nástroji pro blokování reklam a nebude narušovat soukromí uživatelů v podobě
		sbírání uživatelských dat z reklam.
		Nevýhodou o proti reklamám je přidaná náročnost implementace systému umožňující uživatelům přepínat a platit
		plány a hlavně zamezit uživatelům s bezplatným plánem využívat prémiové funkce.
		Avšak na rozdíl od fyzických produktů, kde přidaná zátěž je stálá, tato zátěž se týká především počátku implementace.
		Další možnou nevýhodou je málo uživatelů platící si prémiový plán, protože pokud nebude dostatečný počet
		platících zákazníků, nebude dostatečný příjem pro produkční provoz.

		Konkrétním řešením pro tuto webovou aplikaci bude zavést limity pro bezplatný plán a umožnit uživatelům jednoduše
		přejít na placený.
		Kromě samotných plateb bude možné uplatňovat kupóny, které aktivují prémiový plán na konečnou dobu a poté se plán
		automatické přepne zpět na bezplatný.
		Pokud uživatel budem mít na svém účtě aktivované prémiové funkce při vypršení prémiového plánu, bude nucen
		data zahrnující prémiové funkce odstranit jinak budou automaticky skryta.
		Samotné limity se budou týkat především karet.
		V případě bezplatného plánu bude mít uživatel k dispozici omezený počet karet, které může vytvořit.
		V rámci každé karty pak budou další omezení v podobě nemožnosti tvořit pobočky a zaměstnance.
		Důležité je připravit takový systém, aby bylo možné co nejjednodušeji limity upravovat a případně přidávat další
		s novými prémiovými funkcemi.

	\subsection{Bezpečnost}

	V dnešní době s přibývajícími podvodnými stránkami a sofistikovanějšími útočníky je čím dál více důležité myslet i
	na zabezpečení webových aplikací.
	Kromě základní již běžné praktiky hashování hesel bezpečným algoritmem je nutné zabezpečit mnohem více.
	V první řadě jde komunikaci mezi prohlížečem a serverem, aby nemohl útočník odposlouchávat posílaná data a následně
	se třeba vydávat za určitého uživatele.
	S kradením identity uživatelů souvisí i další praktiky jako CSRF nebo kradení session.
	Z pohledu samotného kódu aplikace je zas nutné připravit systém práv a taková práva uživatelů před vykonáním
	požadovaných akcí ověřovat, aby jeden uživatel nemohl upravovat data jiných uživatelů nebo jinak škodit.

	\subsection{UI}

	Aby byla aplikace schopna konkurovat rozsáhlé konkurenci, musí kromě své základní funkčnosti poskytovat i přehledné
	a vzhledné uživatelské prostředí (\noindent\ac{UI}).
	Navržené \ac{UI} by tak mělo uživateli zjednodušovat přístup k informacím a akcím, které chce provést bez nutnosti
	zdlouhavého hledání skrze aplikací.
	Kvůli velkému procentu využívání internetu na mobilních zařízení je žádoucí připravit prostředí automaticky se
	přizpůsobující velikosti zařízení při zachování přehlednosti a dostupnosti i na malých.
	Hlavním cílem je uživatelům mobilních zařízení poskytnou webovou aplikaci přibližující se svým rozvržením a pohodlím
	nativním aplikacím stejně jako tak jako pro uživatele osobních počítačů s velkými obrazovkami pomocí jedné
	aplikace díky dynamické přizpůsobitelnosti \ac{UI}.
	Tento koncept se navíc v budoucnosti dá dále rozšířit o skutečnou mobilní aplikaci díky některým technologiím
	bez nutnosti tvořit vlastní nativní aplikaci kopírující funkce webové aplikace což je velmi nákladné.

% ################################
\section{Technologie pro vývoj}

Vývoj jakékoliv aplikace od nuly až po provoz v produkčním prostředí zahrnuje spustu kroků a každý takový krok
lze pojmout několika způsoby.
Před samotným vývojem je tak potřeba důkladně promyslet a naplánovat jaké má projekt požadavky, co je cílem samotného projektu
a také jaký je rozpočet pro jeho vývoj.
Od těchto cílů se pak odvíjí způsob vývoje a především výběr technologií a nástrojů použitých při vývoji.
Ty spolu musí být dostatečně kompatibilní a tvořit efektivní celek.
Při výběru nevhodných technologií se totiž může stát, že v průběhu vývoje vývojáři narazí na problém, který je buď
s použitými technologiemi neřešitelný, protože daná technologie třeba nebyla navržena pro konkrétní úlohu, nebo
je nutné problém obejít a to může způsobit komplikace v budoucnu v podobě malé výkonnosti aplikace nebo příliš
složitého zdrojového kódu.
V dnešním moderním světě existuje spousta webových technologií, které umožňují a většinou i značně usnadňují vývoj
webových stránek, aplikací či mikroslužeb.
Avšak nové alternativní technologie přibývají každým dnem a může být mnohdy obtížné i pro zkušené
vývojáře se zorientovat a vybrat ty správné pro daný projekt.
Při rozhodování je důležité vybírat nejen podle popularity, ale především podle typu projektu, ten totiž může zásadně
ovlivnit, jaké technologie jsou vhodné a které nikoliv.

Kromě webových technologií použitých přímo pro vývoj aplikace je vhodné se při návrhu zabývat i technologiemi a nástroji
pro:
\begin{itemize}
	\item uchovávání uživatelských dat
	\item uchovávání souborů a poskytování jejich variant
	\item návrh uživatelského rozhraní
	\item komunikaci s uživateli jinými kanály než je samotná aplikace
	\item běh aplikace ve vývojovém a produkčním prostředí
\end{itemize}

Webové technologie se navíc primárně rozdělují na dvě kategorie: front-endové a back-endové.
Obě obsahují rozdílné technologie zejména kvůli tomu, že se každá z nich zabývá odlišnou částí
vývoje.
Existují ale i technologie spadající více či méně do obou kategorií, avšak v každé plní trochu jiný účel.
Nejrozšířenějším takovým kandidátem je \noindent\ac{JS} dominující v současné době oběma kategoriím.

	\subsection{Techonologie pro návrh UI a UX}

	Návrh \ac{UI} je jedním z nejdůležitějších kroků pro úspěšnou aplikaci.
	Společně s uživatelským zážitkem (\noindent\ac{UX}) z velké části rozhodují zda aplikace nebo webová prezentace
	uživatele zaujme a bude jí nadále používat nebo přejde ke konkurenci.
	Je důležité navrhnout moderní, přehledné a hlavně intuitivní \ac{UI} neobtěžující koncového uživatele svojí složitostí.
	Kromě na vizuálně přívětivého \ac{UI} je stěžejní myslet na \ac{UX}, kdy použité prvky jako tlačítka, nadpisy,
	vyskakovací okna musí být důkladně promyšleny, aby uživatele podvědomě naváděli uživatele k jeho cíli.
	U jednotlivých prvků jde tak především o tvary, barvy, stíny, rozestupy mezi prvky a textem nebo třeba kontrast prvků
	mezi sebou.
	Dobré je myslet i na uživatele s různými omezeními.
	Například pro úplnou slepotu je možné do webové stránky přidat speciální metadata umožňující prohlížečům správně
	předčítat obsah stránky.
	Pro barvoslepost je zas dobré myslet na správný kontrast vybrané barevné palety při použití s textem, aby byl
	čitelný pro všechny.
	Návrh \ac{UI} ještě před samotnou implementací může navíc zásadně šetřit čas vývojáře, protože v úvodních krocích
	se design často mění, a pokud by měl vývojář kromě každou úpravu ještě implementovat, zabralo by to dvojnásobek času.

	Pro návrh \ac{UI} a \ac{UX} před samotným začátkem vývoje nebo i pro dodatečné úpravy existují specializované
	grafické nástroje.
	Použít lze i tradiční grafické nástroje, ty ale nenabízí funkce pro vzájemnou kolaboraci mezi
	členy v týmu nebo přidávání interaktivních prvků přímo do grafického návrhu pro demonstraci fungování budoucí
	aplikace.
	Nejpopulárnějšími nástroji se staly Figma, Sketch a Adobe DX a z toho Figma a Adobe XD poskytují dokonce bezplatné
	varianty s určitými omezeními \cite{design_tools_database}.
	S pomocí nich je snadné navrhnout základní drátový model pro návrh struktury, grafický model pro vizualizaci
	vzhledu i interaktivní demo sestavené z předchozích modelů.
	Tyto modely pak značně usnadňují tvorbu \ac{UI}, protože vývojář tvořící výslednou stránku nemusí vymýšlet
	vzhled a strukturu stránku při samotném vývoji stránky.
	Místo toho se může zabývat čistou implementací až finální verze \ac{UI}.
	Samotný výběr konkrétního nástroje je otázkou podpory operačních systémů designérů a vývojářů, ceny, případně
	předchozí znalosti prostředí daného nástroje.

	% ###############################
	\subsection{Front-endové webové technologie}

	Front-endové technologie se věnují především přímé interakci s uživatelem pomocí webového prohlížeče.
	Definují tak vzhled a samotné ovládání aplikace pro koncového uživatele.
	Tyto technologie pak zpravidla jaksi obalují a zpracovávají funkcionality serverů tak, aby se koncovému uživateli s danými
	daty pracovalo co možná nejjednodušeji a nebyl nucen pracovat se surovými daty.
	V drtivé většině pak neposkytují uživatelům jen data samotná, ale hlavně informace z nich zpracované.

	Tyto technologie by se daly ještě pomyslně rozdělit na standardizované a ostatní.
	Standardizované tvoří základní stavební kameny všech webových stránek a aplikací, kterým rozumí webové
	prohlížeče.
	Ostatní pak představují jakési nadstavby nad těmi standardizovanými a snaží se je nějakým způsobem rozšířit
	či zjednodušit.
	Tyto už standardizované nijak nejsou a mohou se proto často zásadním způsobem měnit od verze k verzi.
	Je jich mnohem více a každým dnem vznikají nové, vývojářům značně ulehčují práci, a proto jsou tolik oblíbené.

		\subsubsection{Standardizované front-endové technologie}

		Jsou definovány standardy společností W3C, což umožňuje vyvíjet v podstatě univerzální stránky a aplikace běžící na
		koncovým uživatelem zvoleném prohlížeči.
		Zástupci těchto technologií jsou HTML, CSS, \ac{JS} a nově i WebAssembly a představují tedy základní
		nástroje pro tvorbu jakýchkoliv stránek a aplikací jež je možné provozovat na internetu. \cite{w3c_webdesign}

		Nejdůležitější je značkovací jazyk \noindent\Ac{HTML}, bez kterého se nelze obejít.
		Tento jazyk definuje základní strukturu a obsah pomocí tzv. značek, které se do sebe zanořují a
		tvoří tak stromovou strukturu bloků dokumentu.
		Struktura vychází z jazyka \noindent\Ac{XML}, který je obohacen hlavně
		o vlastní značky s přidaným významem pro orientaci prohlížečů při vykreslování stránek. \cite{html_hypertext_markup_language}

		Další velmi důležitou technologií je stylovací jazyk \noindent\Ac{CSS} definující vzhled struktury
		dokumentu pro koncové uživatele.
		Pomocí tohoto jazyka je možné upravovat písma, barvy, pozice prvků dokumentu a dokonce i animace.
		Vzhled stránek a aplikací je čím dál tím důležitější jak pro snazší orientaci, tak hlavně pro odlišení se od konkurence.
		\cite{css_cascading_style_sheets}

		V dnešní době už téměř stejně důležitý stavební prvek jako \Ac{HTML} nebo \Ac{CSS} je skriptovací jazyk \ac{JS}
		umožňující na front-endu dynamicky, v průběhu interakce uživatele se stránkou, modifikovat strukturu a vzhled
		původní stránky či aplikace.
		To otevírá obrovské možnosti pro tvorbu komplexních animací, dynamického donačítání obsahu ze serveru, her, přehrávačů
		videí a mnoho dalšího. \cite{what_is_javascript}

		Alternativou případně doplňkem k \ac{JS} je nový standard WebAssembly.
		Je to nízkoúrovňový jazyk podobný Assembleru sloužící jako univerzální jazyk, do kterého je možné překládat
		kód z mnoha již existujících jazyků jako např. C++ nebo Rust.
		Oproti \ac{JS} má ohromnou výkonnostní výhodu, což otevírá možnost pro výpočetně náročné aplikace běžet
		v klasickém prohlížeči.
		WebAssembly se ale dá použít společně s \ac{JS} a lze tak využít výhody obou jazyků. \cite{webassembly}

		\subsubsection{CSS preprocesory, frameworky a knihovny}

		Používání základního jazyka \Ac{CSS} může být zejména u větších projektů mnohdy těžkopádné a špatně udržovatelné.
		Proto existují nástroje jako preprocesory, frameworky a knihovny ulehčující zásadním způsobem práci vývojářům.

		Preprocesory jsou jazyky představující jakousi nadstavbu základního \Ac{CSS} a rozšiřují ho o vlastní
		funkcionality.
		Problém je v tom, že prohlížeče rozumí pouze standardizovanému \Ac{CSS}, proto musí být tyto jazyky překládány
		zpět do \Ac{CSS}, který lze pak použít standardním způsobem.

		Frameworky a knihovny pak seskupují předpřipravené kusy \Ac{CSS} kódu připravené pro rychlé použití při tvorbě stránek.
		Díky nim je možné rychle stylovat dokument stránky, protože vývojář nemusí pomocí \Ac{CSS} konfigurovat vše, co se
		týče vzhledu, ručně.
		Místo toho může použít výše zmíněné kusy kódů nebo již celé hotové bloky.

		Nejrozšířenějším preprocesorem je jednoznačně \Ac{SASS}.
		Oproti základnímu \Ac{CSS} totiž nabízí širokou škálu funkcionalit od vnořování, funkcí pro automatizaci až po třeba
		dědičnost.
		Prohlížeče ale tomuto jazyku nerozumí, proto je nutné ho překládat do \Ac{CSS}, pomocí předpřipravených nástrojů.
		Tento jazyk mimo jiné hlavně usnadňuje organizaci stylovacího kódu a je tak jednodušší ho v budoucnu rozšiřovat a
		modifikovat. \cite{learn_sass}

		Méně používanou alternativou k \Ac{SASS} je \Ac{LESS} a stejně jako \Ac{SASS} rozšiřuje základní \Ac{CSS}
		o další funkcionality jako vnořování nebo dědičnost a je též nutné ho překládat do standardizovaného \Ac{CSS}. \cite{less_overview}
		\Ac{LESS} ale oproti \Ac{SASS} ve spoustě věcech pokulhává a proto není mezi vývojáři tolik oblíbený.
		\Ac{SASS} má např. prokročilejší možnosti scriptování zahrnující cykly a podmínky blížící se programovácím jazykům
		a umožňuje tak jakousi automatizaci generovaného \Ac{CSS} kódu.
		Naproti tomu \Ac{LESS} poskytuje jen základní cykly a proměnné což může být pro některé vývojáře dosti omezující.
		\Ac{SASS} má také mnohem pokročilejší a efektivnější dědičnost, nastavení neduplikuje ale spíše nahrazuje. \cite{sass_vs_less}.

		Dlouholetým favoritem mezi frameworky je Bootstrap.
		Ten obsahuje velké množství základních předpřipravených šablon, nastavení typografie, formulářů, tlačítek atd.
		Kromě toho lze využít komunitních šablon celých stránek a vytvořit finální stránky během velmi krátké doby.
		Bootstrap je ale paměťově náročný a mnohdy těžkopádný, a proto se ho, pokud využívají jen část jeho funkcionalit,
		někteří vývojáři zbavují, a nahrazují ho jednoduššími alternativami či přímo nově vznikajícími standardy
		v samotném \Ac{CSS}. \cite{bootstrap}

		Poměrně novou oblíbenou alternativou k Bootstrapu je TailwindCSS poskytující obecné \Ac{CSS} třídy pro
		pozicování a stylování \Ac{HTML} struktury.
		To umožňuje, stejně jako u Bootstrapu, rychlou tvorbu webových stránek a aplikací bez nutnosti hlubších znalostí
		samotného jazyka \Ac{CSS}. \cite{tailwindcss}
		Oproti Bootstrapu je ale mnohem univerzálnější a díky univerzálním třídám, stránky nevypadají podobně jako v případě
		Bootstrapu, který poskytuje spíše předpřipravené bloky či rovnou celé stránky.
		TailwindCSS je také méně paměťově náročný, a proto nepředstavuje tak velkou zátěž jako Bootstrap.
		\cite{tailwindcss_vs_bootstrap}

		\subsubsection{JavaScriptové frameworky a knihovny}

		Frameworky a knihovny představují v \ac{JS} sadu předpřipravených nástrojů pro snazší a rychlejší vývoj
		oproti čistému \ac{JS}.
		Dříve vývojáři hojně využívali knihoven pro chybějící základní funkcionality, protože nebyl v některých ohledech tak
		pokročilý.
		Dnes je čistý \ac{JS} velmi pokročilý a není problém používat pouze něj, avšak dnes se spíše
		používají frameworky a knihovny pro snadnou tvorbu reaktivních znovupoužitelných komponent.
		Takovéto komponenty představují samostatné bloky, jako např. formuláře nebo tlačítka, udržující si vlastní
		data a při změně se automaticky překreslují.

		Momentálně nejznámější knihovnou pro tvorbu takových komponent je React, jenž
		byl vytvořen především pro tvorbu interaktivních grafických uživatelských rozhraní (\noindent\Acs{GUI}) webových stránek a
		aplikací.
		Zaměřuje se hlavně na zobrazovací vrstvu, což zahrnuje reaktivnost a překreslování komponent a skládání
		výsledné stránky z takových komponent.
		Ostatní funkcionality pak spíše přenechává již existujícím specializovaným knihovnám, které jsou
		často lepší kvůli delšímu vývoji a velké komunitě.
		Stejně jako další podobné frameworky je postaven na systému reaktivity, kdy každá komponenta má vlastní data,
		která když se změní, část stránky se automaticky překreslí s novými daty bez nutnosti explicitního překreslení
		vývojářem.
		Nicméně React může být složitější na naučení a pochopení, kvůli horší dokumentaci nebo rychlému vývoji.
		Rychlý vývoj tohoto frameworku zesložiťuje i samotný vývoj projektů, protože se spoustu věcí často mění a existující
		projekty mohou mít problémy s pozdějšími upgrady na novější verze.
		Má ale velkou komunitu a s tím i spojené velké množství materiálů.
		Navíc je vyvíjen samotným Facebookem, což zaručuje, že se bude ještě dlouhou dobu rozvíjet a jen tak
		nezmizí. \cite{react}

		Další čím dál více oblíbenou alternativou pro snadnou tvorbu interaktivních \Ac{GUI} je Vue.
		Stejně jako React se zaměřuje pouze na zobrazovací vrstvu s využitím reaktivních komponent a ostatní
		funkce přenechává již existujícím knihovnám.
		To umožňuje vývojářům vybrat správné technologie pro daný projekt a nemuset se omezovat vybraným hlavním frameworkem.
		Jeho velkou výhodou je flexibilita míry integrovanosti do projektu.
		Lze ho totiž použít buď jako pomocníka při tvorbě jednoduchých komponent ve stávajícím projektu, nebo jako
		celou platformu pro tvorbu sofistikovaných webových aplikací. \cite{vue_guide}
		Vue je oproti Reactu mnohem jednodušší na naučení, jak pro zkušené vývojáře, tak pro začátečníky
		s programováním, a společně s ne tak často měnícím se kódem je čím dál více používán nejen
		jednotlivci, ale i velkými firmami.
		Nemá sice zatím tak velkou komunitu, ale je dostatečně velká na to, aby se o něm dalo uvažovat
		jako o solidní možnosti pro nový projekt. \cite{vue_vs_react}

		Svelte je nejnovějším přírůstkem do skupiny nejpoužívanějších frameworků pro tvorbu \Ac{GUI}, avšak přináší odlišný
		způsob tvorby interaktivních stránek a aplikací.
		Stejně jako Vue je možné ho použít jen okrajově jako pomocníka nebo pomocí něj vytvořit celou aplikaci.
		Od konkurence React a Vue se ale zásadně liší ve formě v jaké běží v prohlížeči.
		Zatím co React a Vue používají svojí logiku za chodu aplikace k sestavování stránek, Svelte překládá kód
		do čistého kompaktního \ac{JS} kódu.
		To má vliv především na rychlost celé stránky či aplikace, především pak u složitých aplikací,
		ovšem za cenu vyšší rychlosti je zde omezení v podobě vlastního scriptovacího jazyka. \cite{svelte_basics}
		Ten se podobá klasickému \ac{JS}, ale nelze říct, že je s ním zaměnitelný.
		V případě potíží při vývoji tak může být pro některé vývojáře obtížné danou situaci vyřešit. \cite{vue_vs_svelte}

		Mezi tyto frameworky by dal zařadit ještě Angular, který už však ztrácí na popularitě a je používán spíše v
		korporátní sféře.
		Angular je spolehlivý a mocný nástroj a poskytuje obdobné funkcionality jako konkurence,
		nicméně nemá moc dobrou dokumentaci a je obtížný na naučení. \cite{react_vs_angular}

	% ###############################
	\subsection{Back-endové webové technologie}

	Back-endové technologie běží na serverech a proto se na rozdíl od front-endových vůbec nedostanou do styku s koncovými uživateli.
	Místo toho se zaměřují hlavně na práci s daty a informacemi v podobě poskytování relevantních dat front-endovým
	technologiím a provádění mnohdy výpočetně náročných operací nad uživatelskými daty.

	Pro komunikaci mezi těmito dvěma světy se používá standardizovaný aplikační protokol \noindent\Ac{HTTP}.
	Ten definuje strukturu přenášených dat, tak aby jakýkoliv server s jakoukoliv technologií mohl bez problému komunikovat
	standardizovaným způsobem s jakýmkoliv prohlížečem. \cite{http}

	Díky tomuto protokolu servery nejsou omezeny základními standardizovanými technologiemi jako tomu je v případě
	\Ac{HTML}, \Ac{CSS} a \ac{JS}, a vývojáři tak mohou používat teoreticky jakýkoliv programovací jazyk a ekosystém s ním
	spojený.

		\subsubsection{Java}

		Java je velmi univerzální objektově orientovaný programovací jazyk, jenž může být provozován téměř na jakémkoliv
		hardwaru od serveru až po mikrořadiče.
		Umožňuje vyvíjet ohromnou škálu aplikací počínaje desktopovými aplikacemi, přes hry až po třeba právě ty
		serverové aplikace. \cite{java}
		Ačkoliv je psaní kódu v Javě oproti jiným jazykům mnohdy zbytečně zdlouhavé, které nabízejí kratší zápisy
		některých příkazů, Java nabízí velmi stabilní a zpětně kompatibilní prostředí pro vývoj.
		Není se proto třeba obávat, že by se v příští verzi udály velké zpětně nekompatibilní změny nebo dokonce
		by přestala být zcela podporována.
		Java se osvědčila jako vhodná i pro výpočetně náročné serverové aplikace potřebující provádět několik paralelních
		operací najednou.

		V dnešní době jsou serverové aplikace čím dál tím složitější a je v podstatě nutné používat nějaký framework
		či knihovnu, poskytující nástroje pro práci s protokolem \Ac{HTTP} a všeho s tím spojené, pro udržení
		přehledného a snadno rozšiřitelného kódu.

		Jedním z nejpoužívanějších frameworků pro vývoj webových aplikací v Javě je Spring.
		Věnuje se velké škále různých nástrojů pro vývoj a mezi ty hlavní patří systém \noindent\Ac{DI} usnadňující práci s
		mnoho objekty, které jsou mezi sebou propojené a bylo by obtížné tyto vazby spravovat svépomoci.
		Neméně důležitou funkcionalitou je systém událostí poskytující vývojářům možnost jednoduše v rámci programu vyvolávat
		události, na které se lze kdekoliv v kódu napojit a spouštět potřebné procesy. \cite{spring_framework_documentation_core}
		Pro vývoj běžných webových stránek pak poskytuje nástroje pro implementaci \noindent\Ac{MVC}
		architektury, \noindent\Ac{REST} API architektury atp. \cite{spring_framework_documentaiton_web}
		\Ac{MVC} architektura rozděluje zpracovávání požadavku mezi datový model starající se o data a jejich přípravu,
		pohled sestavují výsledné stránky s připravených dat a v poslední řadě řadič zpracovávající požadavky,
		které následně deleguje na datový model a výsledná data předá pohledu. \cite{mvc}
		Čím dál více využívanou architekturou pro komunikace mezi serverem a koncovou aplikací nebo stránkou je \Ac{REST} API.
		Ta se vůbec nestará o to jak se data dostanou ke koncovému uživateli natož v jaké podobě.
		Místo toho se soustředí pouze na zpracování požadavku a poskytnutí relevantních dat, se kterými pak
		většinou pracují front-endové technologie pro sestavení stránek a aplikací. \cite{restfulapi}
		\ac{MVC} a \ac{REST} architektury se tak často kombinují tam kde potřeba ze serveru získat strojově čitelná data
		místo sestavených stránek pro koncové uživatele. \ac{MVC} část se pak stará o zpracování a poskytnutí dat
		a \ac{REST} část se stará do doručení a přijmutí dat.
		Nicméně Spring poskytuje velké množství dalších nástrojů a funkcionalit, a proto může být pro začínající vývojáře
		jeho rozsáhlost odrazující.
		Avšak znalost tohoto frameworku se značně vyplatí, hlavně pak při tvorbě velkých projektů; příkladem mohou být
		třeba e-shopy.

		Alternativou k Springu je Java EE resp. v současné době Jakarta EE (pouze jiný název).
		Ta definuje oficiální standardy pro různé způsoby zpracování \Ac{HTTP} požadavků od nejzákladnějšího obecného
		zpracování, až po např. formátování odpovědi podle daných požadavků.
		Zároveň definuje standardy např. pro mapování objektů na databázové tabulky, \Ac{DI} a atp.
		Tím, že se jedná ve své podstatě pouze o standardy je zapotřebí vybrat pro vývoj knihovny implementující právě tyto
		standardy. \cite{jakarta_ee}

		Obě technologie jsou velmi populární a velmi mocné.
		Každá z nich má své klady a zápory a nelze proto jednoznačně určit lepší z nich, nicméně Spring je obecně jednodušší při
		vývoji a navíc je zdarma.
		Java EE naproti tomu přichází s Oracle licencí, a proto se více hodí pro použití ve velkých korporátních
		společnostech. \cite{java_ee_vs_spring}

		\subsubsection{Javascript}

		V dnešní době je možné scriptovací jazyk \ac{JS} používat nejen k vývoji interaktivního \Ac{GUI},
		ale také k vývoji serverových aplikací zpracovávající \Ac{HTTP} požadavky za pomoci běhového prostředí Node.js
		využívající jádro V8, stejně jako prohlížeče. \cite{express_node_introduction}
		Ačkoliv \ac{JS} běží pouze jednovláknově, je poměrně výkonný pro obsluhu velkého množství menších dotazů díky
		systému událostí, kdy se zpracovává vždy jen ten nejdůležitější požadavek.
		Není ale úplně vhodný pro výpočetně náročné operace, protože pak může blokovat ostatní požadavky, právě kvůli
		neschopnosti rozdělit požadavky mezi více vláken. \cite{js_eventloop}
		Pro některé vývojáře může být použití výhodou, protože v případě použití \ac{JS} též na front-endu,
		nemusí řešit rozdílné jazyky a knihovny.
		Jednoduše použijí jeden jazyk a podobné knihovny.
		Stejně jako v případě Javy existuje mnoho frameworků a knihoven poskytující předpřipravené nástroje pro ulehčení
		tvorby webových stránek.

		Momentálně nejpoužívanějším frameworkem pro back-endový \ac{JS} běžící na Node.js je Express.js. \cite{state_of_js_2020}
		Ten umožňuje především přijímat samotné \Ac{HTTP} požadavky a odesílat \Ac{HTTP} odpovědi a také poskytuje vývojářům
		prostředky pro zpracovávání těchto požadavků pomocí architektury \Ac{MVC}, \Ac{REST} API atd., podobně
		jako Spring v Javě.
		Mimo jiné je pro spoustu ostatních frameworků základním stavebním kamenem, ke kterému pak přidávají
		své funkcionality. \cite{express_node_introduction}

		Next.js je rovněž frameworkem běžícím na Node.js, ale cílí na front-endovou knihovnu React.
		Jeho hlavním úkolem je zjednodušení jeho provozu v produkčním prostředí.
		Může být totiž obtížné nastavit prostředí serveru pro správné poskytování Reactu pro front-end.
		Kromě výše zmíněného jej rozšiřuje např. o schopnost vykreslení částí nebo celků stránek již na serveru a
		prohlížeči posílá připravenou stránku či aplikaci. \cite{create_nextjs_app}

		Obdobou Next.js pro front-endový Vue je Nuxt.js.
		Ten poskytuje dost podobné funkcionality jako usnadnění provozu Vue v produkčním prostředí, sestavování
		částí nebo celků stránek nebo třeba jednodušší podporu pro navigaci mezi jednotlivými stránkami. \cite{nuxtjs}

		Při výběru mezi Nuxt.js a Next.js tak záleží především na volbě front-endového frameworku či knihovny
		nikoliv obráceně, protože nelze použít Nuxt.js společně s Reactem a naopak.

		\subsubsection{PHP}

		Dalším stále velmi oblíbeným, i přes jeho stáří, je skriptovací jazyk \noindent\Ac{PHP} zaměřující se především na tvorbu
		webových stránek a aplikací. \cite{php}

		Tento jazyk je již sám o sobě mocný a dokáže spoustu věcí.
		Avšak poskytuje spíše základní nástroje a vývojáři musí tak psát pokročilejší logiku pokaždé od nuly.
		Proto existují frameworky poskytující jistou úroveň abstrakce vzdávající vývojáře nutnosti tvorby základní logiky,
		jako navigaci mezi stránkami, obsluhu požadavků atp., která je mnohdy pro většinu stránek a aplikací podobná
		né-li stejná. \cite{best_php_frameworks}

		V České Republice je velmi oblíbený český framework Nette díky jeho přehlednosti a univerzálnosti.
		Poskytuje širokou škálu nástrojů a funkcionalit podobně jako Spring pro Javu.
		Též umožňuje vývojářům využít hotového systému \Ac{DI}, implementovat \Ac{MVC} architekturu nebo třeba používat
		vlastní šablonovací systém pro tvorbu zobrazovací vrstvy v rámci \Ac{MVC}. \cite{proc_pouzivat_nette}

		Na poli globálně populárních frameworků dominuje především Laravel.
		Mimo základní funkce přináší schopnost stavět komplexní webové aplikace s velkou rychlostí a bezpečností
		v kombinaci s jednoduchostí vývoje, kterou přinášejí předpřipravené nástroje starající se o běžné repetitivní
		úkoly.
		Kvůli jeho velké popularitě přichází s velkým ekosystémem v podobě např. dostupnosti hostujících serverů
		s instalací na jedno kliknutí nebo velkého množství návodů. \cite{best_php_frameworks}

		Největší konkurencí Laravelu je Symphony, který je starší a má podobně velkou komunitu i ekosystém.
		Narozdíl od Laravelu je rozdělen do komponent podle zaměření funkcionalit a je tak možné je používat samostatně. \cite{best_php_frameworks}
		Symfony oproti Laravelu poskytuje nástroje pro velké projekty jako např. škálování.
		Nicméně se může jevit jako složitější na pochopení pro začínající vývojáře a může být výrazně pomalejší
		než Laravel, kvůli externím knihovnám.
		Naproti tomu Laravel se často mění a může být proto obtížné stávající projekty aktualizovat.
		\cite{laravel_vs_symfony_in_2021}

		\subsubsection{C\#}

		Podobně jako Java je C\# univerzální objektově orientovaný programovací jazyk pro tvorbu nejen
		desktopových aplikací a her, ale právě i webových stránek a aplikací.
		K tvorbě webových stránek existuje oficiální framework ASP.NET, který rozšiřuje základní schopnosti
		komunikace pomocí \Ac{HTTP} protokolu o pomocné nástroje pro snadný vývoj webových aplikací, \Ac{REST} API nebo třeba
		komunikace mezi prohlížečem a serverem v reálném čase. \cite{asp_net}

		Zajímavou a nadějnou novinkou v rámci frameworku ASP.NET je Blazor.
		Ten umožňuje vývojářům vyvíjet i front-endovou interakci s koncovým uživatelem pomocí pouze C\#, \Ac{HTML} a \Ac{CSS}
		bez jakékoliv nutnosti znát \ac{JS}, přičemž Blazor nabízí dvě možnosti, jak tohoto docílit; a to pomocí
		WebAssembly, nebo pomocí již zmíněné komunikace mezi prohlížečem a serverem v reálném čase.
		V případě WebAssembly se kód napsaný v C\# překládá do právě WebAssembly a běží tak čistě v prohlížeči
		jako front-endová technologie.
		Pokud si ale vývojář vybere druhou možnost, v prohlížeči běží jen malý komunikační nástroj delegující operace
		spojené s interaktivním front-endem na server, kde je kód vykonáván přímo v C\#.
		Obě možnosti pak dovolují používat již existující knihovny pro .NET ekosystém, což může být pro některé vývojáře
		obrovskou výhodou stejně jako fakt, že nemusí programovat v \ac{JS}.
		Nicméně vzhledem k tomu, že je tato technologie poměrně nová, nemusí být pro všechny projekty vhodné spoléhat
		se na nezaběhlou a neověřenou technologii, protože není jasné jestli budu ještě dlouhé roky podporována a
		navíc nemá z daleka tak velkou komunitu jako např. React nebo Vue. \cite{blazor}

% ############################

	\subsection{Databázový systém}

	Databázový systém je stěžejní technologie pro uchovávání a agregaci uživatelských dat.
	Bez komplexních databázových systému by nebylo možné efektivně uchovávat strukturovaná data, nad kterými lze vyhledávat potřebné
	informace s možností agregace dat pro různé shrnutí, přehledy a grafy.
	Výběr správného systému je proto jedním z nejdůležitějších kroků před implementací samotného řešení, protože přímo
	zásadním způsobem ovlivňuje jak se s daty bude poté ve výsledné aplikaci nakládat a jaká omezení a výhody vybraný
	systém přináší.

	Existuje spousta databázových systémů s velmi odlišnými zaměřeními a přístupy ke struktuře dat.
	Různé systémy se tak často kombinují, kde každá poskytuje jiná data pro jiné účely.
	Nejčastějším takovým rozdělením u klasických webových aplikací je použití jednoho databázového systému pro
	uchování originálních dat sloužící jako hlavní zdroj dat reprezentující platná data a jako doplněk se pak často vybírá
	systém pro např.: fulltextové vyhledávání agregující pouze nejnovější data z hlavní databáze pro nejoptimálnější vyhledávání.
	Existují samozřejmě i další specializované systémy s různými zaměření jako je mezipamět, vyhledávání podle pokročilých kritérií,
	uchovávání speciálních dat atd.
	Vyvíjená webové aplikace bude však potřebovat pouze uchovávat a vyhledávat data, postačí proto pouze jeden databázový systém
	jako zdroj data a jeden pro vyhledávání.

	Existují dvě kategorie databázových systémů podle způsobu jakým pracují s daty: \noindent\Ac{SQL} a \noindent\Ac{NoSQL}.
	\Ac{SQL} je standardizovaný jazyk postavený na principu relačních databází umožňující pracovat s daty pomocí tabulek.
	Využívá se k manipulaci s databází a uložených dat definující způsob práce s daty v této kategorii systémů.
	Jedná se o velice univerzální nástroj především pro ukládání, vyhledávání a agregaci dat dělající z tohoto jazyka
	nástroj použitelný pro většinu projektů.
	Nicméně i přesto, že je tento jazyk standardizovaný, každý databázový systém postavený na \Ac{SQL}
	má vlastní implementaci a často obsahuje rozšíření nebo změny v syntaxi oproti standardu. \cite{sql_intro}
	Odlišnosti od standardu, ale nejsou příliš velké umožňující uživatelům celkem jednoduše přecházet mezi jednotlivými
	systému bez nutnosti učení se kompletně nového jazyka a vybírat systém spíše podle nefunkčních požadavků než
	podle přístupu k datům.
	\Ac{NoSQL} systémy již nijak standardizované nejsou a každý takový systém obvykle přichází s vlastním
	vyhledávacích jazykem a strukturou dat.
	Naštěstí se během své existence \Ac{NoSQL} systémy kategorizovali čtyřmi široce používanými
	datovými strukturami.
	Nerozšířenější jsou databáze pracující s \noindent\Ac{JSON} či \Ac{XML} dokumenty a databáze využívají strukturu klíč-hodnota.
	V krajních případech se používají i databáze podobné relačním ukládající data v tabulkách nebo databáze pracující
	grafy a jejich uzly. \cite{nosql_explained}

	\Ac{SQL} databáze, jak již bylo zmíněno, ukládají data v tabulkách neboli relacích, kde každý řádek reprezentuje
	jednu entitu/jedince a sloupce reprezentují jednotlivé parametry dané entity.
%	TODO example tabulka
	Jednotlivé relace lze navíc mezi sebou propojovat k vytvoření složitějších struktur pomocí vztahů.
	Díky tomu je možné např.: k entitě navázat kolekci jiných úplně odlišných entit logicky, které spolu logicky souvisí
	nebo tvořit hierarchie entit stejného typu.
	Možnosti návrhu struktury dat jsou tak celkem rozsáhlé, bohužel pro aplikace využívající objektový přístup (\noindent\Ac{OOP})
	je celkem obtížné mapovat tabulky na jednotlivé objekty.
	Při prvním pohledu se může zdát, že objekty a tabulky se dají snadno mapovat, praxe ukazuje mnoho skrytých komplikací
	zmiňované v kapitole o návrhu datového modelu. % todo je to tam fakt zmíněné?

	\Ac{NoSQL} databáze pracující s \Ac{JSON} dokumenty zjednodušují právě zmiňované komplikace při mapování
	na objekty, protože \Ac{JSON} struktura vychází právě z objektové přístupu.
	To zároveň umožňuje jednodušší změnu struktury při změně objektů v aplikaci, ale taky velice ztěžuje agregaci dat
	a vyhledávání na takovými daty.
	Nejjednodušším databázovým systémem je systém využívající strukturu klíč-hodnota kde každá entita má svůj unikátní klíč
	a jednu hodnotu připomínající tabulku o dvou sloupcích.
	Mezi příklady patří: mezipaměť pro rychlé získání konkrétních entit, nákupní košíky nebo třeba preference uživatel.
	Pro analytická data jsou vhodné databáze pracující se sloupci podobně jako relační databáze.
	Na rozdíl od relačních databází jsou ale data uložená ne po řádcích ale po sloupcích což umožňuje jednodušší agregaci
	dat jako je výpočet sumy.
	Tato struktura má ale nevýhodu ve velmi pomalém ukládání, protože vyžaduje několika násobné zapisování dat na disk.
	Posledním rozšířeným typem \Ac{NoSQL} databází je grafová struktura zaměřující se v první řadě na vztahy mezi
	jednotlivými entitami, podle kterých lze vyhledávat jednoduše mezi více entitami na rozdíl od relačních databází, kde
	je potřeba složitě propojovat jednotlivé relace při samotném vyhledávání.
	Tyto databáze se používají většinou spíše jako sekundární databáze pro analýza dat a zkoumaných problémů.
	\cite{types-of-nosql-databases}

		\subsubsection{Databázový systém pro ukládání dat}

		Jak již bylo zmíněno databázové systémy pro ukládání dat se používají jako primární zdroj s nejaktuálnějšími i
		historickými daty.
		U těchto systémů je pak téměř vždy vyžadované pravidelné zálohování dat a schopnost zotavení se z pádu celého
		systému, protože v případě jakékoliv výpadku by mohlo dojít ke ztrátě, často nenahraditelných, dat a to může
		dále vést ke ztrátě uživatelů nebo dokonce zisků.
		Stěžejní je pak schopnost zajistit integritu dat při paralelním zápisu a čtení data, aby nedošlo k situacím,
		kdy každý uživatel vidí stejná data v jiných verzích nebo k vzájemnému přepisu stejných dat.
		To totiž může v lepším případě vést k milným informacím zobrazovaným uživatelům a v tom horším až třeba ke ztrátě
		peněz v případě bankovních systémů pracujících při transakcích se starými nebo porušenými daty.
		Z těchto důvodů existují v databázích transakce definující sled kroků, které musí být splněny při změně dat.
		S tím souvisí množina vlastností zvaná \noindent\Ac{ACID} určující jaké vlastnosti databáze musí
		splňovat, aby se její databázové transakce daly považovat za bezpečné v případě nějakého problému.
		Atomicita definuje, že databázová transakce rozdělená na více kroků musí být splněná celá.
		Pokud alespoň jedna její část selže, selže celá transakce, čímž je zaručena konzistence změny dat v rámci transakce.
		Vlastnost konzistence zaručuje správnost dat oproti pravidlům nastaveným daném datovém modelu, tak aby nebylo možné
		do relací zapsat nevalidní data.
		Izolace transakcí zajišťuje nepřepisování stejných data, pokud dojde k paralelní změně stejných dat ve více
		transakcích.
		Pokud jedna transakce změní nějaká data, ostatní transakce musí počkat až předchozí transakce skončí a až tehdy
		mohou upravovaná data znovu změnit, aby nedošlo k chybnému přepisování dat.
		Aby bylo možné zaručit, že data změněná transakcí, která úspěšně doběhla se nemohou ztratit existuje vlastnost
		zvaná trvalost dat.
		Bez této vlastnosti by nebylo možné se spolehnout, že uložená data v databázi zůstanou. \cite{acid-compliance}

		Nejpoužívanějšími \Ac{SQL} databázemi splňující výše zmiňované požadavky jsou PostgreSQL, MySQL, MariaDB
		nebo třeba Oracle DB.
		Všechny kromě Oracle DB jsou open sourcované a může je využívat kdokoliv bez nutnosti placené licence.
		To neplatí o Oracle DB vyžadující placenou licenci, za kterou uživatel dostane nejen samotnou databázi, ale
		i podporu ze strany firmy Oracle jenž využívají především firmy velkých rozměrů, které mají v takových databázích
		vyšší milióny dat a potřebují proto prvotřídní a přímou podporu ze strany poskytovatele databázového systému.
		Většina firem a jedinců si však toto nemůže dovolit a proto volí spíše ostatní databáze, které poskytují mnohdy
		stejné funkce bez větších omezení v používání díky rozsáhle komunitě přispívající k rozvoji.
		Díky internetovým diskuzím mají také uživatelé těchto databází přístup k diskuzním fórům nahrazující do jisté míry
		placenou podporu.
		Open sourcové databáze poskytují v drtivé většině stejné funkce díky zmiňovanému standardu \Ac{SQL} a
		liší se tak spíše v krajních případech jako jsou podporované datové typy, uživatelská rozšířitelnost systému
		nebo v podporovaných indexes.
		Např.: PostgreSQL obsahuje podporu pro pole hodnot, rozsahy hodnot a pokročilou práci s datovým typem \Ac{JSON},
		na proti tomu MySQL a MariaDB mají omezenější výčet podporovaných datových typů a uzčí schopnosti práce s datovým
		typem \Ac{JSON}.

		Populární alternativou k tradičním \Ac{SQL} databázím se stále více stávají \Ac{NoSQL} databáze
		díky jejich flexibilnějšímu přístupu.
		Zdaleka nejpoužívanější takovou databází sloužící jako zdroj primárních dat je databáze MongoDB využívající
		\Ac{JSON} dokumentů pro ukládání dat.
		Její popularita spočívá především ve snadnosti mapování dat na objekty při zachování \Ac{ACID} vlastností.
		Pro běžné webové aplikace, nezabívající se např.: složitými statistickými výpočty a agregáty, poskytuje MongoDB
		jednodušší a rychlejší vývoj díky podobnosti s objektovým přístupem.
		MongoDB navíc umožňuje větší flexibilitu struktury dat, takže při budoucím rozvoji datového modelu aplikace se
		databáze snadněji přizpůsobí než databáze postavená na \Ac{SQL}.
		Díky tomu jsou databáze \Ac{NoSQL} běžně jednodušeji škálovatelné přes více serverů.\cite{when_to_use_nosql}
		To také ale znamená, že vývojář nemusí chyby spojené se strukturou dat objevit už při vývoji ale až při provozu
		v produkčním prostředí.
		\Ac{JSON} dokumenty mohou také často vést k zvýšené duplicitě dat pokud datový model zahrnuje spoustu
		propojení mezi různými typy entity, protože kolekce \Ac{JSON} dokumentů spolehají na to, že většina dat
		spojená s danou entitou budou obsažena právě v daném \Ac{JSON} dokumentu, a proto podporují jen hodně
		omezené propojování mezi kolekcemi na rozdíl od \Ac{SQL} databází postavených na vzájemném propojování
		relací.
		MongoDB se tak spíše hodí pro aplikace, kde data jsou hodně unikátní a lze tak využít výhod \Ac{JSON}
		dokumentů jako je rychlost načtení nebo škálování.\cite{why_you_should_never_use_mongodb}

		\subsubsection{Databázový systém pro fulltextové vyhledávání}

		Fulltextové vyhledávání umožňuje vyhledávat entity nejen podle přesně daných klíčových slov, ale hlavně podle
		frází nacházejících se v dlouhých textech nebo názvech, které v sobě mohou být zapsány s jiným skloňováním nebo
		dokonce s překlepy.
		Většinou se tak zadaná fráze uživatele vyhledává napříč několika různými texty entit a nalezené výsledky se pak
		řadí podle relevantnosti, tj. jak moc daná entita odpovídá hledané frázi.

		Databáze používané primárně pro vyhledávání dat, a tedy jako spíše sekundární zdroj, nepotřebují nutně pravidelné
		zálohování dat, protože vyhledávací data lze kdykoliv obnovit z hlavní databáze.
		Je spíše kladen důraz na rychlost a dostupnost pomocí duplikace vyhledávacích dat na více serverů do různých částí
		světa pro co nejmenší odezvu.
		Jiná je také struktura dat optimalizovaná pro optimální vyhledávání a nemusí tak obsahovat nevyhledávaná data
		a agregovaná data, která se dají lépe analyzovat vyhledávacím strojem.

		Ačkoliv většina databázových systému použitých jako primární zdroj až už je to PostgreSQL, MySQL nebo MongoDB
		nabízejí podporu fulltextové vyhledávání, vlastnost je to spíše doplňková a oproti dedikovaným databázím
		pro vyhledávání dosti omezující a pomalá.
		Navíc při vyhledávání nad agregovanými daty se tato data musí často duplikovat vedle originálních dat a to přínáší
		nepřehlednost.
		Na proti tomu, při použití dedikované databáze vytvořené speciálně pro vyhledávání, agregovaná data jsou bezpečně
		oddělená od primárních dat a databáze tak může dělat lepší optimalizace.
		Takové databáze navíc poskytují speciální jazyky a nástroje pro specifikování hledané fráze na určitými atributy
		entit bez nutnosti ručního sestavování, mnohdy složitého, vyhledávacího dotazu v tradičních systémech.
		Nejpopulárnější takovou databází je Elasticsearch, který podobně jako MongoDB je \Ac{NoSQL} databází
		využívající \Ac{JSON} dokumenty přinášející snazší mapování na objekty \cite{es_documents}.
		Kromě standardního jazykově specifického fulltextového vyhledávání podporuje například i geolokační vyhledávání
		\cite{es_documents}.
		Čím dál více používanou alternativou je hostovaná databáze Algolia poskytující velmi podobné vyhledávací
		funkce jako Elasticsearch, navíc bez nutnosti složité konfigurace a jednodušším rozhraním pro vývojáře i díky
		předpřipraveným nástrojům.
		Bohužel neposkytuje self-hosting, který umožňoval vývojářům provozovat systém na vlastních serverech, je tak
		nutné využívat servery přímo Algolie s pravděpodobností vyšších nákladů na provoz menší kontrolou nad daty a
		databází.
		Algolia je i přesto pro běžné fulltextové vyhledávání vhodnější díky jednoduššímu rozhraní.
		Na proti tomu Elasticsearch poskytuje pokročilejší nástroje pro složité analytické vyhledávání a možnost
		provozovat aplikaci na vlastních serverech.\cite{es_vs_algolia}
		Relativně novou alternativou přicházející s podobnou jednoduchostí jako Algolia je databáze Typesense.
		Ta je navíc oproti Algolii open-source a lze ji provozovat na vlastních serverech, nicméně momentálně nepodporuje
		statistiky vyhledávání a personalizaci pro doporučování podobných entit.\cite{typesence}
		Problémem u takto nové databáze je pak zejména malá komunita a je tak pravděpodobné, že při implementaci vyhledávání
		nad touto databází vývojář může narazit na problém, na který se mu nemusí dostat řešení a bude odkázán na vlastní
		výzkum a analýzu problému.

	\subsection{Souborové úložistě}

	Je běžné pro dnešní webové aplikace a aplikace obecně uchovávat kromě textových a číselných dat i soubory jako
	obrázky, dokumenty nebo videa.
	Databázové systémy sice umožňují ukládat mezi běžné data i soubory v binární podobě, bohužel je takové řešení dosti
	těžkopádné kvůli zbytečnému narůstání objemu samotné databáze nebo obtížnou manipulaci s takovými daty.
	Při uchovávání souborů se od aplikace očekává, že kromě samotného poskytování souborů uživatelům bude umět tvořit
	varianty obrázků, poskytovat duplicitní úložiště napříč celým světem pro rychlé odezvy nazývané \noindent\Ac{CDN} a zálohování.
	Zejména pak zmíněné varianty obrázků jsou mnohdy tím stěžejním požadavkem, protože umožňují zobrazovat uživatelům menší
	komprimované verze originálních souborů a šetří tak objem dat putující mezi serverem a prohlížečem uživatele rapidně
	zrychlující načítání webové aplikace.

	Souborové úložiště lze implementovat různými způsoby.
	Kromě již zmíněného databázového systému, lze úložiště implementovat ručně nad vlastním serverem, kde se většinou
	provozuje samotná webová aplikace.
	Tento přístup sice vývojáři přináší naprostou kontrolu nad soubory, musí však řešit implementaci tvořiče variant
	nebo pravidelného zálohování svépomoci.
	Musí i pak řešit napojení na externího poskytovatel \noindent\Ac{CDN} systému.
	Existují však služby poskytující tyto funkcionality s minimální konfigurací zbavující vývojáře nutnosti implementovat
	a udržovat vlastní systém společně s otestovanými nástroji poskytovatelem služby.
	Mezi takové služby patří Cloudinary, Sirv, imgix nebo imagekit.
	Všechny nabízejí základní funkce zahrnující tvorbu variant obrázků a systém \Ac{CDN} s jednoduchým napojením
	z webové aplikace.
	Bohužel za cenu jednoduchosti, vývojář přijde o část kontroly nad uživatelskými daty.
	Avšak oproti budování vlastního úložiště je to pro spoustu firem vítaná varianta umožňující především rychlý vývoj
	aplikací.

	\subsection{Technologie pro transakční e-maily a rozesílky}

	Webová aplikace pracující se zaregistrovanými uživateli potřebuje nějakým způsobem komunikovat se svými uživateli i
	mimo samotné prostředí webové aplikace.
	To hlavně z důvodu, pokud se ve webové aplikaci objeví nějaká nová událost, na kterou je potřeba uživatele upozornit
	okamžitě bez čekání až si daný uživatel otevře webovou aplikaci.
	Takovou událostí může cokoliv, zejména pak upozornění na nové přihlášení z neznámého zařízení, nová aktivita spojená
	s daným uživatelem nebo upozornění na nové podmínky používání aplikace.
	Pro tyto a další upozornění se používá převážně emailová komunikace díky její rozšířenosti a univerzálnosti.
	Webová aplikace potřebující komunikovat se zaregistrovaným uživatelem tímto způsobem si musí při registraci
	vyžádat email uživatele.
	Ten je unikátní napříč všemi emailovými poskytovali, a proto ho lze použít zároveň jako unikátní identifikátor
	uživatele využívajícího služeb příslušné aplikace.
	Kromě upozorňovacích emailů webové aplikace čím dál častěji využívají hromadné rozesílky emailů s novinkami
	například nových produktů v případě internetových obchodů.
	Bohužel implementace rozesílek je poměrně složitá kvůli velkému množství odesílaných dat problémů, které mohou nastat
	při doručování jednotlivých emailů.
	Jako příklad lze uvést následující problémy: cílený email může být dočasně zablokovaný nebo
	poskytovatel dané schránky může zamítnout daný email z podezření na spam.
	Vlastníci dané webové aplikace navíc potřebují statistiky o úspěšnosti rozesílek jako počet úspěšně doručených emailů,
	počet prokliků konkrétních odkazů v rozesílaném email apod.

	Podobně jako u souborového úložiště lze takový systém implementovat svépomoci, to však přináší značnou zátěž jak pro
	prvotní implementaci aplikace, tak hlavně na dodatečnou údržbu daného systému.
	Taková zátěž je pak velice znatelná převážně u menších vývojářských týmu.
	Vznikli proto služby zabývající se poskytováním právě těchto služeb s minimální konfigurací pro zbavení se nutnosti
	udržovat další složitý systém.
	Služeb poskytující tyto funkcionality a mnohé další je mnoho.
	Některé poskytují pouze základní infrastrukturu pro odesílání emailů, jiné mají nástroje dokonce pro emailový
	marketing.
	Mezi nejoblíbenější platformy patří Sendinblue, SendGrid, Mailgun, Mailchimp nebo třeba Sparkpost.
	Platformy se liší především svými limity v prodávaných balíčcích.
	Většina z nich obsahuje omezené balíčky zdarma vhodné pro malé webové aplikace nepotřebující vysoké limity odesílaných
	emailů nebo pro vyzkoušení dané platformy.
%	todo nějaké odkazy na zdroje nejpopulárnějších platforem? třeba https://www.isitwp.com/smtp-transactional-email-services/
	% todo z tohodle jsem asi původně čerpal https://www.ventureharbour.com/transactional-email-service-best-mandrill-vs-sendgrid-vs-mailjet/
	% https://www.websiterating.com/email-marketing/mailchimp-vs-sendinblue/
	Při výběru tak hlavně záleží jaké limity jsou pro projekt stěžejní, případně jestli projekt potřebuje nějaké speciální
	marketingové nástroje.

	\subsection{Nástroje pro správu kódu}

	% todo verzovací nástroje a teamové nástroje

	\subsection{Provoz v produkčním prostředí}

	Nedílnou součástí vývoje webové aplikace, u které se plánuje ostrý provoz v produkčním prostředí pro reálně uživatele
	je naplánovaní strategie provozu takové aplikace v produkčním prostředí.
	Oproti lokálnímu vývojovému prostředí, ve kterém aplikace běží v průběhu vývoje na strojích vývojářů, produkční
	prostředí vyžaduje mimo jiné určité bezpečnostní a dostupnostní požadavky.
	Nejdůležitější je zabezpečit produkční server proti neoprávněnému přístupu, tak aby na server mohli přistupovat a
	upravovat jeho vlastnosti pouze administrátoři webové aplikace.
	Dalším bezpečnostním prvek sloužící zejména pro bezpečí koncových uživatelů je HTTPS protokol zajišťující bezpečnou
	a šifrovanou komunikaci mezi prohlížečem a serverem aniž by někdo neoprávněný nemohl odposlouchávat danou komunikaci.
	V dnešní době je tento protokol již standardem a prohlížeče upozorňují uživatele webových stránek nepodporující tento
	protokol.
	Chybějící podpora tohoto protokolu může vést například k odcizení hesel nebo bankovních údajů a útočník se tak
	může vydávat za úplně někoho jiného a způsobit nemalé škody nejen koncových uživatelům ale firmě stojící za danou
	aplikací.
	Šifrování je zajištěno certifikátem SSL podepsaný věrohodnou autoritou spojený s doménou webové aplikace.
	Mimo bezpečnost je stěžejní i nepřetržitá dostupnost aplikace.
	Pokud webová aplikace je často nedostupná pro koncové uživatele, ať už z důvodu zahlcení aplikace požadavky nebo
	chybě v infrastruktuře produkčního prostředí, znamená to, že uživatelé mohou přestat aplikaci využívat a přejít
	ke konkurenci, i v případě že konkurence nemusí nabízet stejné funkcionality.
	To má obrovský dopad hlavně na zisky firmy stojící za takovou aplikací, zejména pak pro internetové obchody jsou
	pravidelní zákazníci klíčoví.

		\subsubsection{Technologie pro provoz produkčního prostředí}

		Existuje několik odlišných variant jak řešit provoz produkčního řešení, každý se svými jasnými výhodami i nevýhodami.
		Nejstarší variantou je pronajmutí serveru.
		Ten lze pronajmou celý nebo jen jeho část ve formě \noindent\Ac{VPS}.
		Pronájem celého dedikovaného serveru je mnohdy zbytečné, pokud aplikace neumí využít celý jeho potenciál.
		Právě díky tomu vznikly zmiňované \noindent\Ac{VPS} poskytující jen takový výkon, který aplikace skutečně využije za
		zlomek ceny.
		Díky tomu lze navíc levněji poskytovat aplikaci z různých koutů světa bez nutnosti vlastnit několik, ne zcela
		využitých, serverů.
		Nicméně takové řešení je velice náročné na konfiguraci a správu kvůli zajištění bezpečnosti a optimálnímu běhu cílové
		aplikace.
		Kvůli tomu provoz vlastního serveru vyžaduje mnohdy dedikovaného správce né-li celý tým, v případě velkých projektu.
		To si však malé start-upy nemohou většinou dovolit a i pro velké organizace je to podstatná finanční zátěž.
		Postupem času a zkušeností s provozem různých webových řešení vznikly různé nástroje pro částečnou automatizaci
		celého procesu, protože podstatná část provozu je ve většině případů repetitivní napříč různými projekty.
		Populárními řešeními pro jednodušší aplikace se stávají předpřipravená konkrétní prostředí pro aplikace využívající
		konkrétní programovací jazyky s jejich ekosystémy.
		Ty umožňují provozovat webové aplikace s minimální konfigurací ze strany administrátora při zachování bezpečnosti
		a dostupnosti.
		Pro náročnější webové aplikace, ať už z důvodu nutné specifické konfigurace nebo spojení více částí tvořící finální
		webovou aplikaci, existují nástroje stojící mezi těmito dvěma řešeními umožňující za cenu nutné malé konfigurace
		relativní volnost ve struktuře dané aplikace bez nutnosti spravovat samotný server.
		Administrátor se pak stará pouze o infrastrukturu dané aplikace a už neřešení infrastrukturu serverů a úložišť.
		Mezi takové nejpoužívanější nástroje patří systémy založené na konceptu kontejnerizace jednotlivých dílčích aplikací.
		Kontejner v tomto kontextu představuje spustitelný balíček spojující kód aplikace a všechny její potřebné závislosti jako jsou
		knihovny, běhové prostředí a konfiguraci prostředí, tak aby bylo možné balíček spustit na jakémkoliv stroji podporující
		běh kontejnerů \cite{what_is_container}.
		Příkladem je systém Kubernetes zajišťující nejen běh několika aplikací najednou a jejich vzájemné síťové propojení,
		ale taky schopnost jednoduše provozovat webovou aplikaci na více serverech najednou což umožňuje distribuci
		uživatelského provozu mezi všechny takové servery, aby nedošlo k přetížení a následné nedostupnosti.
		Alternativou, také postavenou, na kontejnerizaci je Docker, který tento přístup zpopularizoval.
		Velmi oblíbený je pak zejména pro běh webové aplikace ve vývojovém prostředí na vývojářských strojích.
		Avšak lze ho využít i pro provoz produkčního prostředí za cenu nutnosti vetší konfigurace.
		Docker je tak spíše nástroj pro samotnou správu a běh kontejnerů než komplexní správce pro produkční prostředí.
		Síla také spočívá v univerzálnosti kontejnerů.
		Ty mohou obsahovat téměř jakoukoliv aplikaci, přes webové aplikace, desktopové nástroje až po databázové systémy.
		Díky systému jako je třeba Kubernetes je pak možné poměrně jednoduše spustit samotnou webovou aplikace, API server pro
		aplikaci, hlavní databázi a navíc ještě třeba databázi pro vyhledávání pomocí pár příkazů v uzavřené lokální síti
		s pevně definovanými pravidly síťového provozu.

		% todo storage pro docker image
		% todo nějaké zdroje? je to všechno z mojí hlavy tak nevím
		% todo co web servery? nginx apod?

		% todo nemělo by toto být spíš dolejc?
		I přes správně nakonfigurované prostředí však může dojít k chybě například v samotné aplikaci a ta se může stát nedostupnou
		nehledě na to, že samotná infrastruktura funguje v pořádku.
		Pro takovéto případy je dobré mít nějaký monitorovací systém sledující dostupnost aplikace a korektnost odpovědí
		vracených serverem.
		K tomu je většinou využíván strojově ovládaný prohlížeč simulující základní uživatelské kroky při vstupu na stránku
		aplikace.
		Kromě jednoduchého zjišťování dostupnosti, některé nástroje umožňují i monitoring rychlosti načtení stránky pro
		průběžnou kontrolu, že se v aplikaci neděje něco neobvyklého.

		\subsubsection{Poskytovatelé prostředí pro produkční provoz}

		Z důvodu velké náročnosti provozu vlastního fyzického serveru ve vlastní serverovně jednotlivci i menší firmy
		využívají serverů ostatních společností zabývající se čistě provozem serverů k pronájmu.
		Takové společnosti se nazývají poskytovatelé, protože poskytují prostor pro běh jakékoliv aplikace.
		Koncový administrátor se pak už nemusí start o infrastrukturu spojenou s propojením serverů k internetu, a jak už
		bylo zmíněno, mnohdy se nemusí start ani o infrastrukturu vně serveru.

		Mezi nejznámější patří poskytovatelé: DigitalOcean, Microsoft Azure, Google Cloud Platform, Amazon Web Services,
		Linode nebo třeba Vultr. % todo nějaký odkazy? je to čístě z hlavy spíš
		Samozřejmostí je poskytování základních služeb v podobě různých variant \noindent\Ac{VPS} na různě výkonných serverech.
		Lze si ale pronajmou již předkonfigurovaný server přímo pro konkrétní typ aplikace, kdy stačí umožnit poskytovateli
		přístup ke zdrojových kódům a vše ostatní připraví poskytoval.
		Tato varianta je nejjednodušší a ideální pro jednodušší aplikace nemající specifika oproti standardizovanému postupu
		pro vybraný programovací jazyk.
		Není tak ani potřeba mít v týmu administrátora starajícího se pouze o provoz produkčního prostředí.
		Pro složitější aplikace s proprietární konfigurací nebo více dílčími aplikacemi si lze často pronajmout
		předkonfigurovanou strukturu pro běh systému Kubernetes.
		Ani v tomto případě většinou není potřeba speciálního administrátora, protože Kubernetes je navržen především
		pro programátory a v ideálním případě systému říct odkud stáhnout kontejnery s potřebnými aplikacemi.
		Jednotliví poskytovatelé se pak hlavně lišší v cenách za pronájem, kvalitou podpory, dostupným hardwarem, předkonfigurovaných
		prostředích a v garantované dostupnosti.
		Záleží tedy na požadavcích projektu na produkční prostředí a rozpočtu projektu.

		Poskytovatelé samotného prostředí sice většinou umožňují monitorovat aplikaci a samotný server zevnitř serveru,
		pro monitoring aplikace zvenčí existují externí služby zabývající se pouze monitoringem a díky tomu nabízí pokročilé nástroje.
		Takových služeb je nepřeberné množství a většina z nich poskytuje téměř stejné funkce.
		Pravděpodobně nejznámější službou je Pingdom umožňující monitorovat webové stránky z desítek lokalit po celém
		světě \cite{pingdom}.
		Nicméně přichází pouze s placenými plány, které jsou poměrně drahé pro malé a neprofitující aplikace.
		Naštěstí existují omezenější služby poskytující zdarma alespoň základní monitoring dostačující pro pravidelné
		ověřování běhu aplikace i s ověřováním správnosti odpovědi serveru za cenu méně častého provolávání.
		Některé služby dokonce zdarma poskytují pokročilejší funkce jako je třeba SSL monitoring pro kontrolu
		platnosti certifikátů.
		Jako příklad lze uvést Better Uptime, UptimeRobot, HetrixTools nebo Checkly.
		Samozřejmostí jsou pak i placené plány s více funkcemi, takže lze dodatečně podle potřeby funkce získat.
		% todo tady prostě nevím co uvést, všude uvádějí něco jiného, ani nevím kde jsem našel to checkly.
		% prostě asi direct google search
		% todo https://websitesetup.org/website-monitoring-services/
		% todo https://makeawebsitehub.com/best-website-monitoring-services/

\section{Implementace}

Pro samotnou implementaci vlastního navrhovaného řešení je potřeba vybrat správné technologie podle definovaných požadavků
popsaného řešení.
Je dobré se zamyslet nad výběrem technologií z každé kategorie hned na začátku a promyslet, zda všechny vybrané technologie
jsou vůbec kompatibilní mezi sebou, případně zda-li daná kombinace je optimální pro implementovaný projekt.
Při výběru jen části technologií by se mohlo v průběhu vývoje stát, že nějaká nově vybraná technologie nemusí být snadno
zakomponovatelná do existujícího kódu, který je pak třeba nutné z části předělat, což přináší zbytečné další náklady na vývoj.
Po výběru technologií je vhodné nejdříve navrhované řešení načrtnout v nějakém grafickém editoru pro ustanovení finální
podoby a rozvržení struktury aplikace.
Následně je již možné začít se samotnou implementací s pomocí vybraných technologií a prototypu \ac{UI}.

	\subsection{Pojmy}

	Před samotným návrhem implementace je dobré si stanovit a vysvětlit určité běžně používané technické pojmy, které
	nemusí být úplně jednoznačné na první pohled:

		\subsubsection{Business logika}

		Tento pojem označuje hlavní část zdrojového kódu zabývající se naplněním stanovených funkčních požadavků aplikace.
		Běžně taková logika řeší poskytování vyžádaných dat z nějakého úložiště, transformuje data nebo ukládá nová data.
		Může ale i vykonávat složité výpočetní operace, pokud je to jeden z požadavků aplikace.
		Do této části se už ale nevztahuje jak jsou data dotažena z databáze nebo uložena do dní, neřeší komunikace
		s okolím, tj. poskytování \ac{API} reflektující funkcionality business logiky nebo sestavení stránek z poskytnutých
		dat apod.

	\subsection{Návrh implementace}

	Způsobů jak implementovat webovou aplikaci je mnoho a o to těžší může pro nezkušeného programátora výběr být.
	Není totiž vždy úplně jednoznačné jakou cestou se vydat, protože pro většinu běžných webových aplikací je více
	vhodných způsobů.
	Je tak na zvážení implementátora jaký způsob je vhodnější pro daný projekt porovnáním výhod a nevýhod.
	Nejpoužívanějšími jsou následující dva přístupy:

	Starším a do nedávna nejvíce obecně používaným způsobem je serverová aplikace poskytující sestavené stránky
	prohlížeči.
	Taková aplikace sestavuje stránky z programátorem připravených šablon jednotlivých stránek, do kterých se při sestavení
	vloží na určená místa aktuální a požadované informace.
	Těmi může být téměř cokoliv: od jednoduchého údaje jako jméno přihlášeného uživatele přes celý seznam článků k přečtení
	až po kompletně dynamické sestavení dané stránky na základě dat třeba z databáze.
	Pro tvorbu takových šablon existuje velké množství nástrojů a specializovaných scriptovacích jazyků, kde každý
	ekosystém má své.
	Sestavená stránka poslaná prohlížeči je ale víceméně statická, a i když lze takové stránky obohatit například jazykem
	\ac{JS} pro dodatečnou dynamičnost v uživatelově prohlížeči, s rostoucí dynamičností velice roste komplexnost a nepřehlednost.
	Kvůli této nevýhodě společně s nutností tvořit pro každou stránku novou šablonu tento způsob upadá díky lepším novějším
	alternativám.
	Nástroje pro tvorbu šablon sice umožňují do jisté míry přepoužívání částí stránek bez nutnosti duplicity,
	nicméně i tak neumožňují hlavně velkou míru dynamičnosti po sestavení výsledné stránky.
	Výhodou ale je přímé propojení s logikou aplikace a díky tomu snazší přístup k samotným datům bez nutnosti přeposílání
	dat mezi více zařízeními pomocí nějaké standardizované komunikace.
	Pro jednodušší a hlavně méně dynamické stránky v podobě například blogů může být tento způsob stále vyhovující díky
	mírnější křivce učení daných technologií.
	Pro vetší webové stránky a hlavně webové aplikace je tento způsob v dnešní době již zbytečně nepřehledný a složitý a
	neumožňuje takovou dynamičnost jako nativní aplikace.

	Novějším a stále více používaným a oblíbeným způsobem je sestavování stránek pomocí dynamických přepoužitelných
	komponent prvků s pomocí jazyka \ac{JS}.
	Každá komponenta reprezentuje jeden prvek (např.: tlačítko nebo kontejner), který má svoji šablonu pro vykreslení
	a svůj stav zajišťující dynamičnost prvku v čase nebo při interakci uživatele.
	Šablonovací nástroje pro tvorbu komponent jsou velmi podobné těm z předchozího způsobu, nicméně hluboce integrují
	jazyk \ac{JS} pro propojení vzhledu a stavu komponenty.
	Stav takové komponenty může obsahovat pouze například jednoduchý příznak kdy bylo tlačítko stisknuto nebo komplexnější
	data jako uživatelem vložený text, jeho formátování nebo variantu vzhledu.
	Díky zapouzdření stavu a šablony je možné celkem jednoduše takové komponenty používat na více místech bez
	nutnosti přidávat do rodičovské stránky logiku pro ovládání daného prvku.
	Tento způsob pak navíc při využití jeho dynamických schopností poskytnout uživatelům podobný zážitek jako v případě
	nativní aplikace.
	Pro zajištění všech těchto vlastností se používá jazyk \ac{JS} běžící v prohlížeči, kde se veškeré sestavování běžně
	děje.
	Nicméně prohlížeč tímto způsobem nemá přístup k datům a je proto nutné zajistit ještě nějaký zdroj dat a komunikaci
	mezi tímto zdrojem a aplikací v prohlížeči.
	Tím je většinou serverová aplikace poskytující \ac{API} zpřístupňující data pomocí standardizované strojové komunikace,
	které musí rozumět jak serverová aplikace tak i aplikace v prohlížeči.
	\ac{API} je obecný název pro rozhraní poskytující více aplikacím možnost strojově komunikovat mezi sebou pomocí
	strukturovaných dat.
	Momentálně nejpoužívanějšími přístupy pro tvorbu \ac{API}
	Aplikace vystavující \ac{API} definuje jaké data a funkce umí zpracovat a ostatní aplikace při doržení těchto pravidel
	mohou získávat z této služby data nebo je naopak odesílat ke zpracování.
	Momentálně nejpoužívanějšími přístupy jsou \ac{REST} \ac{API} a GraphQL \ac{API}, kde každý má své výhody, nevýhody
	a myšlenku jak takové \ac{API} má vypadat, nicméně slouží ke stejnému účelu: poskytovat data frond-endové aplikaci
	ze serveru.
	I tak je oproti předchozího způsobu, kde šablony mají přímí přístup k aktuálním datům, složitější správně synchronizovat
	informace mezi prohlížečem a serverem právě z důvodu separace, kde si front-endová aplikace musí hlídat data
	korektně sama data aktualizovat.
	Především je tato nevýhoda viditelná při implementaci systémů uživatelů a přihlášení.
	Jak už se může na první pohled zdát, tento způsob tvorby aplikace je složitější, protože vyžaduje znalosti více technologií
	najednou a schopnosti všechny tyto prvky správně propojit do fungujícího celku.
	Další nevýhodou jsou zvýšené nároky na výkon cílového zařízení uživatele kvůli zmíněné dynamičnosti probíhající přímo
	v uživatelově prohlížeči.
	Z pohledu vyhledávačů je zde nevýhoda omezených možnostech nastavení \ac{SEO} vlastností kvůli posunutému sestavování
	v prohlížeči s čímž má většina botů prohledávající weby pro problém kvůli snížené podpoře \ac{JS}.
	Naštěstí zvýšenou náročnost překračují již popsané výhody v podobě velké dynamičnosti, přehlednosti a znovupoužitelnost.
	Navíc již existují i nástroje sestavující prvotní podobu stránek již na serveru a díky tomu lze poskytnout sestavenou
	stránku uživateli dříve a se správným nastavením pro \ac{SEO}.

	I když by se tedy dali použít pro implementované řešení oba způsoby, bylo vybráno řešení využívající dynamické komponenty
	zejména z důvodu požadované velké dynamičnosti aplikace přibližující se nativním aplikacím.
	Tento způsob navíc umožňuje budoucí transformaci webové aplikace na téměř nativní mobilní aplikace díky systému
	sestavování stránek přímo v prohlížeči podobně jako nativní aplikace jsou sestavovány přímo v zařízení.
	Původní prototyp aplikace byl sice implementován pomocí prvního zmíněného způsobu za pomocí šablon stránek, jak se
	ale ukázalo krátce po začátku vývoje, tento způsob by byl spíše limitující při snaze vyvinout moderní dynamickou aplikaci.
	Bylo proto rozhodnuto pojmout výzkum dostupných a vhodných technologií sofistikovaněji a začít víceméně od znovu. % todo možná přesunout někam do úvodu a třeba rozšířit že to bylo dobré rozhodnutí a potvrzuje to nutnost výběru technologií dopředu
	Znamenalo by to také vyvinout vlastní ekosystém pro poskytnutí dynamičnosti zbytečně oddělený od samotných stránek,
	což dále vede k nepřehlednosti a horšímu rozšiřování z důvodu chybějícího zapouzdření komponent.
	Místo toho by bylo nutné udržovat separátně šablony stránek a obecnou logiku pro dynamičnost starající o místo
	jednotlivých komponent o vše.
	Kromě jasných výhod pro tento konkrétní projekt je zde ještě vedlejší benefit v podobě zkušeností získaných
	z návrhu komplexnějšího systému, kde jednotlivé části mezi sebou musejí korektně komunikovat.

	% todo reactivní aplikace??? tady v případě api serveru

	Při implementace řešení využívající serverové \ac{API}, je zásadní vybrat jaký přístup bude implementovat.
	Tento způsob totiž neurčuje specificky jak má být taková aplikace implementovaná s jakými konkrétními nástroje, jde
	spíše osvědšený vzor, který je dobré následovat a tím že \ac{API} není nějaká specifická technologie ale také spíše
	způsob jak tvořit komunikaci, je potřeba se zamyslet nad konkrétním způsobem před samotným výběrem technologií.
	Pomineme-li kompletně vlastní řešení, které by bylo pracné na vyvinutí a nepřinášející žádné benefity pro tento projekt,
	je na výběr mezi přístupy \ac{SOAP} (Simple Object Access Protocol), \ac{REST} nebo GraphQL \ac{API}.
	Vzor \ac{SOAP} je již poměrně starý a i když nabízí rozsáhlé možnosti jeho struktura založená na značkovacím jazyku
	\ac{XML} je zbytečně složitá oproti alternativám a příliš upovídaná, což může zbytečně zatěžovat a zpomalovat
	síťovou komunikaci.
	Z tohoto důvodu již nějakou dobu výměně strukturovaných dat dominuje jazyk \ac{JSON}, který je značně stručnější a
	čitelnější a hlavně jednodušší na zpracování.
	Na tomto jazyka staví právě vzory \ac{REST} a \ac{GraphQL}.
	\ac{REST} staví na principu poskytování jednotlivých ucelených dat v předem specifikované formě a objemu v
	podobě jednotlivých "resource", které lze získat nebo zapsat a to vše bez návaznosti mezi jednotlivými požadavky
	\cite{restfulapi}.
	GraphQL je novější a o proti \ac{REST} postaven na myšlence získání jen těch dat, které aplikace opravdu potřebuje díky svému
	speciálnímu dotazovacímu jazyku a pevná struktura dat je tak spíše udává jaké všechna data lze získat a jak budou vypadat
	pokud si o ně aplikace skutečně požádá \cite{graphql}.
	GraphQL tak řeší problém se získáváním příliš mnoho dat nebo nedostatek dat s nutností dělat několik separátních dotazů
	pro získání všech potřebných dat díky možnosti specifikování jaká data jsou vyžadovaná oproti získání
	přesně daných dat z \ac{REST} \ac{API}.
	Nelze ale říci, že GraphQL je za všech podmínek lepší než \ac{REST}, protože každé \ac{API} musí z pohledu serveru
	řešit stejné problémy v podobě serializace dat, zpracování chyb, cachování a mnoho dalšího.
	GraphQL tak sice poskytuje lepší kontrolu nad vracenými daty, nemusí však být vhodné pro některé projekty (např.:
	obecné mikroslužby poskytující \ac{API} pro kohokoliv), navíc klient konzumující takové GraphQL \ac{API} potřebuje
	komplexnější knihovnu pro zpracování takových dat.
	Kromě toho, \ac{REST} za svou delší existenci vyspěl a existuje nepřeberné množství dostupných materiálů jak navrhnout
	správné \ac{API} a nástrojů pro samotnou tvorbu.

	Právě kvůli vyspělosti \ac{REST} \ac{API} a rozšířenosti padlo rozhodnutí právě na tento přístup.
	Vzor \ac{REST} je navíc stále velmi populární a používaný a tak hlubší znalosti návrhu \ac{API} pomocí tohoto vzoru
	nejsou k zahození.
	Obecně jsou pro tento projekt vhodné oba přístupy a ani jeden by nekomplikoval vývoj, protože každé mají své výhody.

	\subsection{Výběr technologií}

	Po rozhodnutí jakým způsobem bude projekt fungovat je možné vybrat konkrétní technologie vhodné pro vybraný způsob,
	pomocí kterých bude aplikace vyvíjena.
	I přes to je patrné, že vybrat správnou technologii pro daný projekt není snadný úkol, především pokud vývojář
	nemá ani okrajové zkušenosti s většinou těchto technologií a nemůže tedy vybírat podle vlastních poznatků.
	Pokud jsou ale vybrány celosvětově používané a podporované technologie, nelze většinou udělat chybu výběrem ani
	jednou z nich, protože každá je většinou dostatečně univerzální.
	Krajní výjimkou můžou být specifické projekty s náročnými požadavky splňující pouze užší výběr technologií.
	Tam je pak oproti běžnějším projektům, jako jsou webové stránky a aplikace, kladen větší důraz na výběr správných
	technologií, protože některé technologie nemusí poskytovat dostatečnou podporu pro dané specifické požadavky a tím
	se může vývoj značně zkomplikovat.
	Kromě prodloužení implementace se totiž může stát, že je potřeba podstatnou část aplikace kompletně předělat.
	Nicméně pro běžnější projekty hlavní rozdíl pro většinu vývojářů spočívá ve stylu, jakým se v dané technologii
	vyvíjí a jestli daný vývojář má alespoň nějaké znalosti používaného programovacího či skriptovacího jazyka.
	Čas strávený studováním nových technologií v porovnání s konkurenčními technologiemi, které již vývojář zná, může být
	poměrně velká část samotného vývoje.

		\subsubsection{Kritéria pro výběr technologií}

		Jak již bylo zmíněno, vyvíjený projekt bude webovou aplikací potencionálně dostupnou ve všem moderních prohlížečích.
		Stěžejním kritériem je vybraný přístup s využitím dynamických komponent a tudíž aplikace bude rozdělena
		na front-endovou aplikaci a \ac{API} server.
		Toto kritérium zásadním způsobem omezuje možnosti vybratelných technologií, protože ne všechny webové technologie
		a programovací jazyky tento přístup podporují a mají pro něj připravené nástroje.
		Důležité je i omezení pouze na moderní prohlížeče, protože pokud by byl požadavek podporovat i několik let staré,
		již nepodporované prohlížeče používající jen malé procentu uživatelů, možnosti výběru technologií by to značně
		ovlivnilo a omezilo.
		Nebylo by možné použít novější technologie značně usnadňující vývoj a bylo by nutné dělat mnohem více kompromisů
		při výběru technologií a především při samotném vývoji.
		Webová aplikace, jak již bylo zmíněno, se bude zabývat možností tvorby, zveřejňování a vyhledávání
		karet obsahující veřejně dostupné a vyhledatelné informace, např.: odkazy na sociální sítě,
		kontaktní údaje nebo popis daného subjektu.

		Hlavním požadavkem je vytvořit webovou aplikaci, která se bude chováním a interakcí co nejvíce přibližovat
		k nativní aplikaci jak pro osobní počítače, tak pro mobilní zařízení zároveň (podle typu zařízení, na kterém aplikace
		v danou chvíli poběží) bez nutnosti instalace takové aplikace, aby byl poskytnut co možná nejrychlejší přístup
		k informacím v dané aplikaci.
		Kromě odpadající nutnosti instalovat aplikaci tento přístup přináší i značné urychlení vývoje v porovnání s
		nativními aplikacemi.
		Pro každou platformu by totiž musela být tvořena separátní aplikace od začátku vícekrát znovu a znovu.
		Výjimkou jsou programovací jazyky a frameworky umožňující tvorbu opravdových aplikací běžících na několika
		platformách zároveň, podobně jako webové aplikace.
		I tak se ale tyto aplikace musí instalovat a distribuovat separátně.

		Dalším velmi důležitým aspektem pro aplikaci poskytující spoustu vyhledatelných informací je možnost nastavit
		kvalitní SEO, aby se daná aplikace dostala do předních žebříčků při vyhledávání uživateli skrze internetové
		vyhledávače.
		\noindent\Ac{SEO} je tedy technika pro optimalizaci webových stránek pro internetové vyhledávače jako např.
		Google, Bing nebo DuckDuckGo, tak aby byly schopné co nejlépe pochopit a kategorizovat informace poskytnuté
		danou stránkou a předat dané informace koncovému uživateli. \cite{what_is_seo}

		\subsubsection{Vybrané technologie}

			\paragraph{Návrh UI a UX}

			Prvním nástrojem při návrhu aplikace, jak již bylo zmíněno, měl být grafický editor pro náčrt struktury a vzhledu.
			Vzhledem k tomu, že všechny dostupné nástroje obsahují podobné základní funkce, byl vybrán editor Figma díky
			schopnosti běhu v prohlížeči a tím odpadající nutnosti vlastnit zařízení s operačním systémem Windows nebo macOS.
			Program je navíc bezplatný i pro základní kolaboraci mezi členy v týmu v případě potřeby.
			Figma je tak ideální volbou i pro začátečníky bez zkušeností ostatních grafických editorů.

			\paragraph{Technologie pro vývoj GUI}

			Vzhledem k tomu, že cílem je vytvořit webovou aplikaci, která poběží v standardních webových prohlížečích,
			nelze se úplně vyhnout základním front-endovým technologiím \Ac{HTML}, \Ac{CSS} a \ac{JS}.
			Ty lze ale do jisté míry nahradit jejich nadstavbami.

			V případě \Ac{CSS} byl vybrán preprocesor \Ac{SASS}, pro jeho širokou komunitu a funkce.
			Tento preprocesor umožní přehlednější kód a tím rychlejší vývoj.
			Kvůli výše zmíněné potřebě zobrazovat různé typy odkazů na profilových kartách, \ac{SASS} navíc umožní
			do jisté míry automatizovat repetitivní a nepřehledné \ac{CSS} třídy pro velké množství barevných kombinací
			(každá sociální síť používá jiné barevné kombinace ve svém designu).

			V případě tvorby interaktivního \Ac{GUI} je výběr složitější.
			Je možno využít jak základního \ac{JS}, tak frameworků či knihoven usnadňující tvorbu reaktivních komponent.
			Alternativou může být jazyk C\# s pomocí technologie Blazor, díky kterému je možné se úplně od ekosystému
			\ac{JS} odstínit za cenu menší komunity a vyspělosti.
			Ze studijních účelů byl vybrán \ac{JS}, především pro hlubší pochopení tohoto jazyka, protože je používán v současné
			době téměř každou webovou stránkou či aplikací a pravděpodobně tomu tak ještě dlouhou dobu bude.
			Jako nadstavba jazyka pro zrychlení vývoje byl vybrán framework Vue díky jeho velmi dobré dokumentaci, snazšího
			pochopení a možnosti programovat přímo v jazyku \ac{JS}.
			Vue (a stejně tak ostatní takové nadstavby) sám o sobě ale nedisponuje zrovna dobrými prostředky pro \Ac{SEO},
			což je dáno delegací sestavení výsledné stránky až na front-endový \ac{JS}.
			Kvůli tomu vyhledávače při strojovém procházení takových stránek nevidí výslednou stránku a nemohou tak
			vyextrahovat potřebné informace.
			Řešením může být vykreslení takové stránky již na serveru a prohlížeči poslat již připravenou stránku, kterou
			následně převezme samotný Vue.
			Tuto techniku poskytují již hotovou serverové frameworky Nuxt.js nebo Next.js.
			Protože byl ale vybrán Vue, lze vybrat pouze Nuxt.js mající tuto techniku jako jednu z jeho hlavních výhod.
			Kromě lepších nástrojů pro \ac{SEO}, sestavení stránky na serveru zrychlí prvotní vykreslení stránky
			koncovým uživatelům, protože daná stránka přijde ze serveru již sestavená.

			\paragraph{Technologie pro vývoj API}

			Co se týče technologií pro vývoj serverové aplikace fungující jako \ac{API} poskytující data, tam je výběr spíše
			otázka preferencí a předchozích znalostí daného vývojáře s programovacími jazyky a frameworky.
			Všechny zmiňované totiž nabízí alespoň základní nástroje pro tvorbu \ac{API} aplikace.
			Některé však nabízí i komplexní podporu pro tvorbu \ac{API}.
			Tím, že byl již vybrán serverový framework Nuxt.js pro podporu \Ac{SEO}, by bylo možné jednoduše použít Node.js,
			na kterém Nuxt.js běží v kombinaci např. s frameworkem Express.js a vystavit \Ac{REST} \ac{API} přímo pomocí něj.
			Ekvivalentní alternativou pro tvorbu samotného \Ac{REST} \ac{API} je použití např. jazyka Java, C\#, \ac{JS}
			nebo \Ac{PHP}.
			Ze studijních důvodů a vývojářovy předchozí znalosti byl vybrán právě jazyk Java společně s frameworkem
			Spring poskytující pokročilé nástroje potřebné pro tvorbu jednoduchého i komplexního \Ac{REST} \ac{API} a jeho
			automatizované testování bez nutnosti řešit licenční poplatky jako v případě Java EE.
			Kromě podpory pro \ac{REST} \ac{API}, ekosystém Spring přichází například s vlastní knihovnou pro správu
			zabezpečení dat a komunikace pomocí práv a různých doporučených taktik v různých oblastech pro znemožnění zcizení
			identit a podobně.
			Knihovna je má navíc taktéž podporu pro \ac{REST} \ac{API} a díky tomu lze jednodušeji zabezpečit celou aplikaci.
			I když je knihovna poměrně rozsáhlá a náročná na pochopení, bezpečnost webu je mnohem složitější a při vývoji
			kompletně vlastního řešení bezpečnosti se mnohem jednodušeji může stát, že vývojář zapomene na podstatnou část a ohrozí tak
			data uživatelů o proti použití knihovny řešící všechny tyto problémy systematicky a často automaticky.
			Není však problém knihovnu kdykoliv celkem jednoduše rozšířit o vlastní požadavky, dělající z knihovny
			nenahraditelný nástroj.

			% todo asi by bylo fajn tady zmínit i ORM, ale db se řeší až dál
			% možná i další stěžejní knihovny?

			\paragraph{Databázový systém jako zdroj dat}

			Pro uchovávání uživatelských dat je nutné vybrat vhodný databázový systém.
			Pro účely tohoto projektu by se daly použít jak \ac{SQL} databáze tak i \ac{NoSQL} databáze.
			Projekt nevyžaduje specifické náročné analytické agregace dat, které by bylo možné poskytnou pouze nějakou z
			\ac{SQL} databází, dalo by se tak uvažovat o \ac{NoSQL} databázi MongoDB pro ulehčení mapovaní objektů na
			\ac{JSON} dokumenty.
			Nicméně navrhované řešení počítá s určitou hierarchií karet, což už není by v případě MongoDB mohl být značný
			problém, pokud jednotlivé karty, i když v hierarchii, musí být dostupné na stejné úrovni pro koncové uživatele
			\cite{why_you_should_never_use_mongodb}.
			Projekt bude sice vyžadovat zprvu pouze poměrně jednoduchou jednoúrovňovou hierarchii, i tak by ale docházelo ke
			zbytečně duplicitě dat kvůli požadavku na ukládání i podřízených karet do oblíbených nebo požadavku a na vyhledávání
			všech karet pomocí jednoho vyhledávače.
			Z těchto důvodů padla volba na některou z \ac{SQL} databází.
			Mezi bezplatnými databázemi by se dala vybrat například PostgreSQL, MySQL nebo MariaDB a všechny byly schopny
			poskytnou potřebné funkce pro vývoj navrhovaného řešení.
			Proto finální výběr byl tak spíše preferencí podle programátorových předchozích znalostí s bezchybnou funkčností,
			dostupnými funkcemi a vyspělostí databázového systému PostgreSQL.

			\paragraph{Databázový systém pro fulltextové vyhledávání}

			% todo

			\paragraph{Souborové úložiště}

			Dalším důležitým úložištěm dat je úložiště uživatelských souborů.
			V tomto případě se jedná především o obrázky a možnosti jednoduché tvorby jejich variant pro optimalizaci \ac{UI}.
			Jak již bylo zmíněno v analýze, možností s největší kontrolou nad daty by bylo vyvinou vlastní řešení s pomocí
			různých knihoven.
			Takové řešení by bylo ale poměrně pracné a i přes velký investovaný čas by nemuselo poskytovat všechny funkce
			poskytující ostatní řešení.
			Protože cílem tohoto projektu není vyvinou optimální souborové úložiště ani uchovávat přísně tajná data, je
			vhodnější využít hotového řešení a vývojové prostředky investovat do samotné aplikace.
			Vhodné řešení by tak mělo poskytovat kromě \ac{CDN} hlavně tvořič variant.
			Takových služeb je vícero a mnohdy jsou svými poskytujícími funkcemi ekvivalentní a výběr je tedy opět spíše
			preferencí.
			Vybrána byla služba Cloudinary především díky její rozšířenosti mezi ostatními uživateli poskytující všechny
			požadované funkce pro tento projekt a navíc bezplatně.

			\paragraph{Transakční emaily a rozesílky}

			Protože aplikace bude pracovat se zaregistrovanými uživateli bude potřeba uživatelům odesílat transakční emaily
			o důležitých událostech spojených se změnou v daném účtu, případně o jiných důležitých událostech jako změna podmínek
			aplikace a podobně.

			Transakční emaily jsou tedy emaily odesílané konkrétnímu uživateli připravené na míru danému účtu, nikoliv
			hromadné rozesílky v podobě reklam nebo obecných novinek aplikace.
			Opět jako v případě souborového úložiště lze připravit vlastní implementaci zpracovávání šablon emailů a jejich
			odesílání pomocí knihovny JavaMail starající se pouze o samotné odeslání a doručení emailu.
			Avšak u emailů kromě samotného odesílání řešit dynamické šablony (podobně jako u šablon stránek) a připravit
			takovou infrastrukturu, aby emaily skutečně byly doručovány a nepadali do spamu.
			Na proti tomu většina nástrojů zmíněných v analýze poskytuje poměrně jednoduché \ac{API} pro odesílání emailů
			s již připravenou infrastrukturou pro tvorbu šablon a jednoduchou možnost nastavit odesílání emailů tak aby
			byly správně doručovány.
			Na druhou stranu bezplatné plány jsou omezené na počet možných odeslaných emailů za nějakou dobu a za větší limity
			se musí logicky platit.
			Avšak přináší tato řešení záruku určité spolehlivosti doručitelnosti emailů díky používaným serverům s dobrou pověstí
			u emailových serverů, které jen tak daný email nezahodí do nevyžádané pošty.
			Navíc pro případ speciálních požadavků na správu emailů aplikací existují služby jako je SparkPost zaměřující
			se pouze na spolehlivé doručení emailů bez nadbytečných nástrojů.
			V případě této aplikace je důraz především na jednoduchost tvorby šablon a odesílání transakčních emailů bez
			speciálních požadavků.
			To umožňuje využít zvolenou služby SendinBlue poskytující intuitivní editor šablon emailů, jednoduché \ac{API}
			pro odesílání emailů s parametry pro šablony a navíc podporuje i rozesílky v případě budoucího rozvoje.

			\paragraph{Provoz aplikace}

			Poslední částí skládanky je samotný provoz ve vývojovém lokálním prostředí, produkčního prostředí a případně
			i v dalších, například: testovacích prostředí.

			Jak již bylo v analýze nástrojů a poskytovatelů zmíněno, možností jak webovou aplikaci provozovat je mnoho: od
			manuálního nastavování a spouštění až po částečnou automatizaci v podobě kontejnerů.
			Protože se jedná o nový projekt nemá moc smysl se vydávat starou metodou manuálního nastavování a spouštění všeho.
			Místo toho se vyplatí jít cestou kontejnerizace, protože umožňuje jednoduché nahození podpůrných systémů
			(např.: databáze), na kterých aplikace stojí, a zároveň umožňuje připravit izolované kontejnery samotné aplikace
			podle vyvinutých verzí což poměrně zjednodušuje celý proces sestavovaní aplikace ze zdrojového kódu a přenos
			spustitelné aplikace na cílené prostředí mimo lokálního.
			Jako bonus je možné jednoduše spouštět historické verze aplikace (např.: pro ověření nějaké funcionality nebo chyby)
			bez nutnosti transformovat zdrojový kód do nějaké
			předchozí verze a následného sestavení pro samotné spuštění.
			Staré verze navíc mohou obsahovat jiné a zastaralé postupy sestavování dané aplikace a nemusí být jednoduché
			přenastavit vývojové prostředí pro tuto skutečnost.
			Kromě zjednodušeného přenosu mezi prostředími se zjednoduší i přenos aplikace i mezi členy případného týmu.
			Nevýhodou je znalost dalšího nástroje pro sestavování takových kontejnerů a jejich následné spuštění, u většiny
			projektů je ale tato nevýhoda mnohonásobně překročena zrychlením a zjednodušením vývoje.

			Pro jednoduché projekty lze využít ještě jednodušší variantu v podobě sdílení zdrojového kódu s poskytovatelem
			prostředí a poskytoval sám automatizovaně zajistí sestavení a spuštění aplikace na cílovém prostředí.
			Avšak taková aplikace musí splňovat určitou standardizovanou strukturu a neumožňuje spuštění více aplikací tvořicí
			jeden systém.
			Navíc vývojář přijde o schopnost jednoduše spouštět historické verze aplikací a zjednodušeného sdílení mezi členy
			v týmu.

			V rámci tohoto projektu je potřeba spouštět dvě aplikace: front-endové \ac{GUI} a serverové \ac{API}.
			Tyto části spolu musí umět komunikovat a tvářit se navenek jako celek.
			Právě na tento scénář je vhodné využít kontejnerů a nějakého orchestrálního nástroje jako Kubernetes.
			Tento nástroj umožní poměrně jednoduše čitelně definovat jaké aplikace v jakých verzích se mají spouštět, jaká mají
			práva a jak mezi sebou budou po sítí komunikovat.
			Je požadavkem aby \ac{GUI} a \ac{API} mezi sebou mohli komunikovat pouze v rámci interní sítě uvnitř jednoho serveru
			a pro koncové uživatelé zobrazoval pouze funkční celek v podobě interaktivního \ac{GUI}.
			Kubernetes s tímto přístupem počítá a automaticky zajišťuje potřebnou interní komunikaci a v základu žádnou komunikaci
			nepustí ven ani dovnitř serveru.
			Je pak na vývojáři, případně administrátorovi, aby explicitně definoval jakým způsobem bude systém komunikovat s
			okolním světem pomocí jednoduchých pravidel.

			Výběrem systému Kubernetes pro provoz částí implementovaného systému zužuje výběr možných poskytovatelů pouze na
			nutné minimum.
			Sice v dnešní době již většina poskytovatelů nabízí servery Kubernetes, nicméně často lišší cenou daného řešení.
			% todo nějaké srovnání cen
			Mezi dlouhodobě nejlevnější mezi nejoblíbenějšími patří poskytovatelé DigitalOcean a Linode.
			Oba poskytují široké možnosti konfigurace, přívětivou cenu, rozsáhlou dokumentaci, základní monitoring hardwaru
			a ochotnou podporu.
			Z vlastních zkušenosti byla dána přednost DigitalOcean díky jednoduší konfiguraci celého Kubernetes systému
			zejména pak při nastavování bezpečnostních funkcí a celkového propojení s danou Kubernetes konfigurací běžících
			webových aplikací.
			DigitalOcean navíc jeden z nejpopulárnějších poskytovatelů a tak kromě oficiální dokumentaci a podpory lze
			najít spousta komunitních materiálů.

			Finálním nástrojem je monitorovací systém automatizovaně kontrolující produkční prostředí jestli aplikace
			stále běží a je dostupná pro koncové uživatele.
			Těchto externích nástrojů je celá řada a většina těch co poskytují bezplatné plány nabízí podobné funkce a
			a konkrétní volba je tak spíše otázkou preferencí nebo předchozích zkušeností s nějakou z dostupných služeb.
			Pro tento konkrétní projekt byla vybrána služby Checkly díky poměrně široké nabídce funkcí, zejména pak
			možnost monitorovat webové dotazy, dotazy přímo na \ac{API}, expiraci SSL certifikátu pro šifrovanou
			komunikaci a možnost upozorňovat programátora či správce pomocí emailu nebo komunikačních aplikací
			(např.: Discord), což je vhodné zejména pro vývojové týmy, kde případné upozornění vidí všichni.

	Jak je vidět, výběr konkrétních technologií je poměrně náročný úkol, zejména pak pokud programátor nemá praktické
	zkušenosti s vývojem různých typů aplikací a technologií, aby dokázal spolehlivě určit vhodné technologie před samou
	implementací.
	Avšak pořádná analýza před samotných vývojem může několika násobně zrychlit vývoj, i pokud se některé, ne tolik
	stěžejní, technologie v průběhu vývoje změní.

	\subsection{Prototyp}

	Prvotní verze aplikace byla sice taktéž postavená na frameworku Spring společně \ac{ORM} frameworkem Hibernate pro
	přístup k databázovému systému MySQL.
	Tato verze stavěla na starším způsobu tvorby webových aplikací se sestavováním jednotlivých stránek již na serveru pomocí
	šablon.
	Neexistovalo tak rozdělení na front-endovou \ac{GUI} aplikace a \ac{API} server aplikace předávající si data mezi sebou
	pomocí \ac{REST} \ac{AC}.
	Místo toho existovala pouze jedna standardní webová aplikace poskytující již sestavené stránky pomocí dat z databáze.
	Aplikace však zahrnovala střípky \ac{REST} \ac{API} se snahou udělat \ac{GUI} více dynamické např.: při editaci poznámek
	oblíbených karet.
	Aplikace tak měla vlastně dvě části: hlavní kde byly šablony a přístupové třídy k datům hluboce propojeny a vedlejší
	poskytující omezená data pomocí \ac{API}.
	Nicméně i přes to že se jednalo o rannou fázi aplikace s minimem funkcí, začalo být postupně znát, že vytvořit
	plně dynamickou aplikaci chovající se podobně jako nativní aplikace by bylo při nejmenším kostrbaté.
	Pro rozšíření \ac{API} by se navíc samotná business logika musela k tomu přizpůsobit a být více modulární.
	Šablony stránek navíc byly tvořeny bez jakéhokoliv vzhledu \ac{UX} návrhu a obsahovaly pouze nejnutnější prvky pro
	interakci.
	To však pro aplikaci vyžadující intuitivní \ac{UI} není vhodné a mohlo by se stát, že při implementaci finálního vzhledu
	by se podstatná část jednak šablon, ale samotných stránek mohla zásadně změnit což by vyžadovalo velký zásah pro již
	existujícího kódu.

	Právě kvůli těmto obavám a možnostech moderních technologií byl kompletně změněn přístup k vývoji a bylo potřeba začít
	od nuly s analýzou technologií a možností.
	Část původního kódu se sice přepoužila, nicméně původní kód byl po analýze natolik přepracovat, že se dá uvažovat
	spíše o nové aplikaci.
	Nově tedy bylo rozhodnuto řádně zanalyzovat využívané styly a postupy vývoje webových aplikací společně s populárními
	technologiemi pro dané styly.
	Následně bylo potřeba navrhnout alespoň prototyp \ac{UI} a \ac{UX} pro lepší orientaci a explicitního definování
	funkcionálních požadavků a způsobu interakce uživatele s aplikací.
	Po této komplexní analýze a návrhu bylo již možné začít implementovat přepracovanou verzi aplikace.

	\subsection{UI a UX}

 	Zejména pro explicitní definování funkcionálních požadavků a způsobu interakce uživatele s aplikací byl před samotnou
	implementací věnován čas návrhu právě \ac{UI} a \ac{UX}.
	Tento krok navíc umožnil *něco(zobrazit)* vizi finální aplikace a zhodnotit designová rozhodnutí přímo v grafickém
	editoru.
	Právě grafický editor Figma napomohl zhotovit téměř kompletní grafický návrh jednotlivých prvků a stránek a uceleně
	definovat designový jazyk budoucí aplikace.
	Díky tomuto návrhu bylo možné postupně vypilovat grafické prvky, styly a strukturu na jednom místě díky systému
	znovu použitelných komponent, kdy se změny promítly do celého návrhu.
	Bylo tak možné lépe nahlédnout na designový jazyk jako celek použitý v různých scénářích a upravit ho tak, aby byl
	dostatečně univerzální pro co možná nejširší skupinu úloh.

	Kromě samotného rozvržení struktury a vytvoření nadhledu nad aplikací, právě příprava a vypilování designového jazyka
	bylo jedním s velkých benefitů návrhu \ac{UI} v grafickém editoru.
	Designový jazyk je soubor pravidel, schémat a stylů grafického designu, kterým se řídí vývoj veškerých prvků \ac{UI}.
	Tento jazyk má především zajistit jednotnost designu napříč celou aplikací pro poskytnutí intuitivnějšího rozhraní,
	tak aby uživatel nemusel stejnou věc dělat kompletně jinak v různých částech aplikace.
	To by totiž přineslo značnou frustraci koncovému uživateli což by mohlo vést úbytku využívání.
	Z tohoto důvodu je nutné věnovat návrhu designového jazyka nějaký čas a zamýšlet se nad všemi možnými scénáři používání
	prvků.

	% TODO DL z dropboxu
	Design jazyk si bere velkou inspiraci \href{https://material.io/}{Material Design 2} definující jakýsi soubor
	základních rozvržení, velikostí prvků, barev a podobně.
	Nicméně navržený design jazyk se v některých prvcích rozchází kvůli potřebám tohoto projektu a není proto pravidlem,
	že Material Design 2 striktně kopíruje, spíše slouží jako výchozí bod pro nové prvky, které ještě nebyly navrženy
	design jazyku.
	I tvůrci Material Designu berou tyto materiály jako výchozí bod a počítají s tím, že daný projekt si ho přizpůsobí
	tak aby co nejlépe vystihoval danou aplikaci.
	Z Material Designu si bere hlavně \href{https://material.io/design/layout/spacing-methods.html#baseline-grid}{velikosti a odsazení}
	mezer a prvků a základní pravidla jako pozicování tlačítek, rozvržení seznamů, modální okna a podobně.

		\subsubsection{Rozvržení prvků a grafická pravidla}

		Rozvržení prvků by mělo být maximálně přehledné a intuitivní.
		Mezi hlavní pravidla patří nezatěžování stránek příliš mnoha informacemi a zobrazit koncovému uživateli pouze to
		co je skutečně nutné.
		Pokud je nutno zobrazit velké množství informací, je lepší rozdělit jednu stránku na více přehlednějších, aby
		uživatel nebyl přehlcen a neztratil lehce pozornost.
		Další podstatným pravidlem je zakomponování tzv. bílého prostoru kolem prvk.
		Bílý prostor znamená prostor kolem prvků oddělující jednotlivé prvky od sebe prázdnotou.
		Prvky tak poté nejsou na sobě tak nalepené a uživatel se dokáže lépe orientovat a užívat si používání aplikace.
		Neméně důležité je vizuálně navést uživatele co na dané stránce může vše provést bez zbytečných překážek a
		složitostí.
		Díky tomu je pravděpodobnější že uživatel dosáhne svého cíle bez frustrace, opuštění aplikace nebo vyhledání
		pomoci.
		To také může velice přispět k návratnosti uživatele do aplikace.
		Pro intuitivnost prostředí a spokojenost uživatele je dále nutné udržovat konzistenci grafických pravidel a prvků
		napříč celou aplikací, aby uživatele jen tak něco nepřekvapilo s čím by se neuměl poradit.
		Konzistence zaručí efektivnější a příjemnější práci s danou aplikací.
		\cite{create_great_ux}

		Zajímavou a užitečnou pomůckou pro návrh rozvržení prvků je princip vizuální rovnováhy.
		Vizuální rovnováha staví na pomyslené vizuální váze prvků, kde určité vlastnosti, jako konstrast nebo složitost,
		určují jakosi váhu prvku.
		Následně dva prvky porovnává na fyzikální páce, která musí zůstat v rovině i pokud jsou prvky jinak těžké pomocí
		posunu jednotlivých prvků od sebe.
		% todo citace obrázku https://uploads-ssl.webflow.com/5c72b8b72d05ce02c7987c46/5d066cd3d9b514108395ac6f_naprava-rovnovahy-1.png
		\includegraphics[width=0.24\linewidth]{obrazky/rovnovaha_paka.png}\hfill
		Pokud je tedy nejaký prvek vizuálně těžší než druhý, je dobré lehčímu prvku přidat více bílého prostoru a odsunout
		ho tak od těžšího prvku.
		Tím vznikne rovnováha mezi prvky a celkově přívětivější rozhraní.
		\cite{vizualni_rovnovaha}

		Rozvržení prvků ale není vše a je potřeba myslet i na samotný design.
		Je potřeba definovat jasná pravidla a styly, aby byla zajištěna konzistence.

		Nejdůležitější částí designu je správně připravená paleta barev.
		Bylo potřeba připravit spolu kompatibilní barvy použitelné napříč aplikací.
		Hlavní část palety obsahuje obecné barvy: hlavní prezentovanou barvu použitou na místech, která musí být
		patřičně zvýrazněna a barvy pro různé texty jako hlavní text, nadpis, podnadpis nebo popisek.
		Dále paleta obsahuje paletu obecných základních barev pro případ například uživatelských úprav některých objektů.
		Důležitými barvami jsou i barvy pro označování polí formuláře, zejména pak jejich správnosti při validaci
		hodnot.
		Paleta obsahuje i ostatní méně obecné barvy a je samozřejmě možné ji v budoucnu rozšiřovat, avšak je potřeba se
		vždy zamyslet jestli je nová barva skutečně nutná a zda-li je kompatibilní s již použitými barvami.
		Většina barev je inspirována výše zmíněným Material Designem obsahující širokou paletu barev, kterou lze jednoduše
		využít v jakémkoliv projektu.

		\includegraphics[width=0.24\linewidth]{obrazky/paleta_barev.png}\hfill
		Navržená paleta barev. % todo anotace

		Dalším velmi důležitým designovým prvkem jsou stíny pomáhající vizualizovat pomyslnou vzdálenost jednotlivých
		prvků od uživatelova zraku.
		Díky stínům je totiž možné u více překrývajících se prvků jasně ilustrovat uživateli, které prvky jsou více v
		popředí než ostatní.
		Tím lze zároveň poměrně efektivně odlišit jednotlivé bloky od sebe.
		Způsobů používání stínu je mnoho.
		Pro tento projekt byl vybrán způsob, kdy stín se snaží simulovat reální stín generovaný jedním světelným zdrojem.
		Na takové stíny jsou uživatelé zvyklí z reálného světa a je pro ně jednodušší pochopit oddělení prvků.
		Velmi používané je i využití linek kolem jednotlivých prvků pro oddělení bloků, případně se kombinuje obojí.
		Ve skutečnosti i tento projekt používá kombinaci linek a stínů.
		Stíny jsou v tomto případě použity hlavně pro bloky ucelených informací a prvků které mají vyšší prioritu než
		ostatní v rámci jednoho bloku a linky jsou použity pro standardní prvky uvnitř bloků.
		Kombinace byla vybrána z důvodu zjednodušení vzhledu, protože použití pouze stínů by mohlo pro některé uživatele
		být příliš rozptylující.

		\includegraphics[width=0.24\linewidth]{obrazky/ukazka_stinu_a_linek}\hfill
		Ukázka kombinace stínů a linek v jednom bloku. % todo anotace

		Kromě linek a stínu je dobré myslet i na samotné okraje prvků, tj. zda-li budou mít zaoblené rohy a podobně.
		Po vzoru Material Designu byly zvoleny zaoblené rohy o průměru 8px pro větší lehkost prvků.
		Toto rozhodnutí je však celkem subjektivní a někteří uživatelé mohou preferovat ostré rohy.

		\includegraphics[width=0.24\linewidth]{obrazky/blok_obsahu}\hfill
		Ukázka zaoblených rohů. % todo anotace

		Posledním pravidlem je jednotná signalizace výběru.
		Tato část musí být obzvlášť konzistentní napříč aplikací a intuitivní pro koncové uživatele.
		Uživatel musí být schopen jednoznačně určit jaký prvek ze všech možných je vybraný ať už se jedná o navigaci
		nebo výběr položky ve formuláři.
		Velice používaným způsem jak tuto signalizaci řešit je jednoduchá linka zobrazená u vybraného prvku.
		V rámci této aplikace byly navrženy dvě varianty: varianta pro pás prvků a varianta pro samostatné prvky.
		Varianta pro pás prvků se použije v případě pásu prvků, u kterých nemůže dojít k přetečení na více řádků.
		V takovém případě se použije jednoduchá linka pod či nad prvkem.
		V případě osamocených prvků nebo víceřádkových prvků je použita linka kolem celého okraje prvku.

		\includegraphics[width=0.24\linewidth]{obrazky/ukazka_vyberu_1}\hfill
		Ukázka signalizace výběru v navigaci. % todo anotace

		\includegraphics[width=0.24\linewidth]{obrazky/ukazka_vyberu_2}\hfill
		Ukázka signalizace výběru konkrétního prvku. % todo anotace

		Na těchto principech je tedy možné navrhnout nějaké základní univerzální prvky pro tvorbu konkrétních stránek
		a následně celý vzhled.

		\subsubsection{Strukturální prvky}

		Strukturálními prvky jsou myšlené takové prvky, které slouží jako základní stavební prvky pro jednotné
		rozvržení stránek a oddělení prvků uceleným způsobem.
		Tyto prvky tak slouží jako prvky obalovací a bez vnořených interaktivních prvků nedávají samostatně smyl
		a jsou především pomůckou pro tvoření stránek.

			\paragraph{Blok obsahu}

			\includegraphics[width=0.24\linewidth]{obrazky/blok_obsahu.png}\hfill

			Stránky jsou rozvrženy hlavně pomocí tzv. bloků obsahu označující jeden logický celek informací
			na stránce.
			Blok obsahu vždy obaluje nějaký konkrétní obsah: dlouhý text, seznam položek, editor a cokoliv dalšího.
			Blok obsahuje hlavičku obsahující nadpis popisující logický celek informací, co se bude v bloku nacházet
			a k čemu slouží, a lištu tlačítek dávající smysl pro celý logický celek seřazenou zprava od nejdůležitějších
			akcí.
			Samotný obsah těla už je pak záležitostí konkrétního použití, avšak pravidlem je jednotné odsazení 24px od
			okrajů bloku.
			Krajním případem je využití bloku bez hlavičky na speciálních stránkách s vlastním nadpisem, kde blok
			představuje logický celek informací platných pro celou stránku (příkladem mohou být stránky s popisem
			funkcí aplikace uživateli).

			V případě více logických celků na jedné stránce se použije více bloků řazených pod sebou a oddělených 24px
			pro dostatečné oddělení rozlišných informací.

			\includegraphics[width=0.24\linewidth]{obrazky/blok_obsahu_ukazka.png}\hfill
			Ukázka využití bloku % todo styl anotace obrázku

			Uvnitř bloku se může nacházet jakýkoliv obsah.
			Nicméně je několik základních prvků, které se zde běžně nacházejí a je nutno aby byly konzistentní.
			Jedná se především o seznamy prvků, textová pole a vnitřní logické celky.
			Seznamy prvků definují vlastní odsazení a proto jsou výjimkou nepoužívající standardní vnitřní odsazení
			bloku.
			Dobrým pravidlem je omezit množství seznamů v jednom bloku na jeden, protože více seznamů může představovat
			značné zmatení uživatele.
			Textová pole vlastní odsazení nemají, nicméně pokud takových polí je v jednom bloku více, musí být oddělena
			mezi sebou 16px a horizontálně zobrazovat maximálně dvě pole.
			Ve některých případech je nutné logický celek rozdělit ještě na dílčí celky, většinou pomocí speciálních
			tlačítek odkazujících na jiné stránky.
			Ty po vzoru bloků mají stín pro odlišení od běžných prvků a mezi sebou mají prostor 16px stejně jako
			textová pole.

			\paragraph{Lišta akcí}

			\includegraphics[width=0.24\linewidth]{obrazky/lista_akci.png}\hfill

			Kromě hlavní navigační lišty aplikace dostupné téměř na všech stránkách pro jednoduchou navigaci mezi
			částmi aplikace, bylo nutné zavést ještě jeden typ lišt: lišta akcí.
			Díky ní je možné na určitých stránkách zobrazit sekundární lištu s nadpisem a dodatečnými akcemi vztahující
			se k obsahu stránky jako celku.
			Takovými stránkami jsou většinou stránky zobrazující detail určitého jednoho objektu, který lze například
			smazat nebo upravit, ale zároveň je potřeba zobrazit vnořená data (příkladem může být editor jedné karty).
			Navíc může obsahovat i sekundární navigaci spojenou s konkrétním objektem.

			\paragraph{Modální okno}

			\includegraphics[width=0.24\linewidth]{obrazky/modalni_okno.png}\hfill

			Modální okna představují bloky obsahu překrývající všechen ostatní obsah při zobrazí, aby se uživatel zaměřil
			pouze na obsah uvnitř okna a nebyl rušen zbylým obsahem, který v danou chvíli není důležitý.
			Jedná se především o potvrzení určité akce vyvolané uživatelem nebo zobrazení detailu nějakého dílčího prvky.
			Modální okno se může zobrazit pouze jako reakce na uživatelskou akci, nesmí dojít k samovolnému zobrazování
			náhodných oken, protože to vede k frustraci uživatele.
			Každé okno má hlavičku a případně i patičku, avšak obsah se mění podle typu okna.
			Tato aplikace rozlišuje dva primární typy oken: informační okno a akční okno.
			Informační okno slouží pouze k zobrazení již existujících dat a nevyžaduje od uživatele žádnou akci.
			Lze ho zavřít buď křížkem v hlavičce nebo kliknutím mimo okno a neobsahuje patičku.
			Akční okno akci vyžaduje a uživatel musí vždy nějakou akci zvolit, nelze okno zavřít pouhým kliknutím mimo okno.
			Takové okno má v hlavičce pouze nadpis a obsahuje patičku se všemi možnými akcemi, které uživatel může provést.
			Akce jsou většinou dvě a jsou seřazeny zprava od té nejdůležitější.
			Příkladem takových akcí může být potvrzení vytvoření objektu a zrušení akce.
			U tohoto typu okna se navíc předpokládá, že bude obsahovat kromě statických i nějaký formulář, který musí
			uživatel vyplnit, není to však pravidlem.
			Tlačítko nejdůležitější akce by mělo být patřičně odlišeno od běžných tlačítek, aby bylo zřejmé co se po
			uživateli chce.

			Okna mají podobná pravidla odsazení jednotlivých vnitřních prvků jako bloky obsahu.
			Obsah by měl být odsazen 24px od okrajů okna (až na výjimky s vlastním odsazením) a akční tlačítka jsou
			mezi sebou odsazeny 8px.
			Celá patička je navíc od samotného obsahu taktéž odsazena 24px pro jasné rozdělení částí.

			\includegraphics[width=0.24\linewidth]{obrazky/modalni_okno_informacni_ukazka.png}\hfill
			Ukázka konkrétního informačního modální okna. % todo anotace

			\includegraphics[width=0.24\linewidth]{obrazky/modalni_okno_akcni_ukazka.png}\hfill
			Ukázka konkrétního akčního modálního okna. % todo anotace

			\paragraph{Kontextuální nabídka}

			\includegraphics[width=0.24\linewidth]{obrazky/kontextualni_nabidka.png}\hfill

			Jedná se malé menu zobrazující rozšířené možnosti nějaké akce.
			Většinou se používá pro zobrazení vícero akcí možných nad nějakým objektem, pokud není v \ac{UI} dostatek
			místa pro všechny akce nebo pro výběr konkrétní položky výběrového pole ve formuláři.
			Nabídka podobně jako modální okno překrývá část obsahu pro zviditelnění, nicméně ji lze jednoduše zavřít
			kliknutím mimo nabídku.

			\includegraphics[width=0.24\linewidth]{obrazky/kontextualni_nabidka_seznam_ukazka.png}\hfill
			Ukázka kontextuální nabídky akcí položky v seznamu. % todo anotace

			\includegraphics[width=0.24\linewidth]{obrazky/kontextualni_nabidka_vyberove_pole_ukazka.png}\hfill
			Ukázka kontextuální nabídky položek výběrového pole. % todo anotace

			\paragraph{Sekce}

			\includegraphics[width=0.24\linewidth]{obrazky/sekce.png}\hfill

			Pro stránky složené z více podstránek (např.: nastavení uživatelského účtu) je vhodné provázat tyto stránky
			nějakou navigací pro jednoduchý přechod mezi dílčími stránkami.
			K tomu slouží extenze bloku obsahu o postranní menu sloužící právě pro navigaci souvisejících stránek.
			Díky viditelnému menu je jasně viditelné jako sekce jsou uživateli dostupné a co tak uživatel může vše dělat.

		\subsubsection{Prvky pro interakci}

		Po prvcích strukturující obsah je možné navrhnout již prvky, se kterými bude uživatel přímo či nepřímo interagovat.

			\paragraph{Tlačítko}

			Nejhlavnějším takovým prvkem je tlačítko.
			To umožňuje uživatelům vyvolávat akce, potvrzovat akce nebo se nechat přesměrovat na jiné stránky.
			Tlačítek je v aplikaci spousta a každé dělá trochu něco jiného.
			Lze je ale zařadit do pár kategorií a navrhnout tak pár typů tlačítek evokující jejich prioritu a vážnost akce,
			se kterou jsou spojeni.
			Všechny tlačítka byla rozdělena na tlačítka s nízkou prioritou, tlačítka s vysokou prioritou a tlačítka s vysokou
			prioritou a vysokou vážností akce.

			Tlačítko s nízkou prioritou reprezentuje vedlejší akci, kterou sice uživatel může provést, ale není hlavním cílem.
			Takové tlačítko má neutrální barvu a ve většině případech neobsahuje ani ikonu, aby nevyjadřovalo vysokou prioritu
			a uživatel se zaměřil spíše na výraznější tlačítka.

			\includegraphics[width=0.24\linewidth]{obrazky/tlacitko_s_malou_prioritou}\hfill
			Ukázka tlačítka s malou prioritou. % todo anotace

			Tlačítko s vysokou prioritou představuje primární akci, kterou by se uživatel měl vydat.
			Taková tlačítka představují potvrzení nějaké akce, vytvoření nových objektů, přidání existujících objektů a
			podobně.
			Měli by být proto patřičně zvýrazněna a uživatele podvědomě navést kam má kliknout pokud v danou akci chce
			provést.
			Tato tlačítka mají proto výraznou barvu a vystihující ikonu pro zpřehlednění a zpříjemnění celkového dojmu.

			\includegraphics[width=0.24\linewidth]{obrazky/tlacitko_s_velkou_prioritou}\hfill
			Ukázka tlačítka s vysokou prioritou. % todo anotace

			Pro případy kdy uživatel provádí nějakou akci s vysokou vážností jako je například nevratné smazání nějakého
			objektu, vzniklo rozšíření tlačítka s vysokou prioritou.
			Tlačítko se lišší pouze v červené barvě evokující dojem závažné akce.
			Právě červená barva by měla uživatele přimět znovu zamyslet se nad prováděnou akcí, protože červená barva
			je využívána jako symbol něčeho negativního.

			\includegraphics[width=0.24\linewidth]{obrazky/tlacitko_s_velkou_prioritou_a_velkou_vaznosti}\hfill
			Ukázka tlačítka provádějící nevratnou změnu. % todo anotace

			\paragraph{Vstupní pole}

			Vstupní pole různých druhu jsou společně s tlačítky nejdůležitějšími prvky jakékoliv aplikace.
			Vstupní pole umožňují uživatelům předat data aplikaci v různých formátech podle požadované informace.
			Vzniklo proto několik základních typů vstupních polí, které lze jednoduše kombinovat uvnitř v formulářů.
			Mezi ně patří: textové pole, výběrové pole, pole pro výběr času a data, pole pro nahrání obrázku a pole pro
			výběr geografické lokace.
			Vstupní pole využívají linek namísto stínu, protože sami o sobě nereprezentují celek ale pouze část celého
			bloku.
			Pole mají popisek pro identifikaci účelu konkrétního pole, ikonu a mohou mít i akční tlačítko pro jednoduchou
			akci nad danou zadanou hodnotou.
			U většiny polí je navíc popisek plovoucí, kdy pokud pole nemá žádnou hodnotu, nahradí ji popisek, pokud však
			nějaká hodnota byla zadána, popisek se zmenší a posune stranou, aby zůstal stále viditelný, ale nepřekážel.
			Pole jsou také schopna změnit svoji barvu a zobrazit popis chyb, pokud je zadaná hodnota nevalidní.
			Pokud je hodnota validní nebo žádná ještě neexistuje, je možné na místo chyby pod pole vepsat nápovědu k danému
			poli pro koncového uživatele v případě nejasných polí nebo potřebě dodatečných informací v podobě například
			požadovaného formátu.

			Textové pole slouží pro zadávání různých krátkých i dlouhých textů, mají vždy popisek a mohou mít ikonu a
			akční tlačítko.
			Mohou sloužit i pro zadávání hesel a v takovém případě zobrazují pouze zástupné znaky.
			Obsahují však tlačítko pro přepnutí mezi zástupnými znaky a originálními pro kontrolu zadaného hesla.

			\includegraphics[width=0.24\linewidth]{obrazky/textove_pole}\hfill
			Ukázka textového pole s ikonou a nápovědou. % todo anotace

			Extenzí textového pole jsou pole pro zadání času a data.
			Ty rozšiřují manuální zadávání o okénka pro pohodlný výběr bez nutnosti klávesnice.

			\includegraphics[width=0.24\linewidth]{obrazky/datumove_pole}\hfill
			Ukázka pole s výběrem data a času. % todo anotace

			Vzhledem k požadavku možnosti specifikovat geografické lokace karet bylo potřeba zavést i specifické pole
			pro výběr bodu v mapě.
			Pole zobrazuje automaticky mapu světa, kterou je možné hýbat a přibližovat ji a vybírat konkrétní body.
			Vybraný bod lze také jednoduše smazat/resetovat.
			Na rozdíl od textového pole toto pole neumožňuje specifikovat ikonu ani akční tlačítko a obsahuje pouze
			statický popisek kvůli stále zobrazené mapě do všech krajů.

			\includegraphics[width=0.24\linewidth]{obrazky/pole_lokace}\hfill
			Ukázka pole pro výběr geografického bodu v mapě. % todo anotace

			Posledním méně používaným polem v této aplikaci je pole pro výběr obrázku.
			Pole umožní výběr souboru ze souborového systému uživatele, který je následně nahrát a zobrazen jeho náhled
			přímo v poli.
			Vybraný obrázek je pak možné jednoduše změnit nebo úplně odstranit.

			\includegraphics[width=0.24\linewidth]{obrazky/obrazkove_pole}\hfill
			Ukázka pole s nahraným obrázkem. % todo anotace

			\paragraph{Seznam položek}

			Velmi využívaným prvkem je seznam jakýchkoliv položek.
			Seznam může být jednoduchý pouze pro jednoduché zobrazení několika podobných položek nebo může umožňovat
			spravovat objekty stejného typu pomocí přidělených akcí a dodatečných informací.
			Samotný seznam je pouze obalovací prvek, až prvky konkrétních položek obsahují pokročilé možnosti.
			Každá položka může obsahovat ikonu nebo obrázek, text, seznam akcí nebo dokonce i vnořený seznam dílčích
			položek stahujících se k dané nadřazené položce.
			Nicméně vnořené seznamy jsou omezeny pouze na jednu úroveň kvůli zachování přehlednosti, více úrovní navíc
			běžně není potřeba a u takového scénáře by bylo na zvážení zda-li je vícenásobné vnoření tím správným řešením.
			Seznamy akcí se kromě běžného zobrazení umí i seskupit v případě malého místa a schovat akce do kontextuální
			nabídky, aby uživatel využívající malé zařízení nepřišel o možnost využívat všechny akce s minimálním omezením.

			\includegraphics[width=0.24\linewidth]{obrazky/jednoduchy_seznam}\hfill
			Ukázka jednoduchého seznamu podobných prvků. % todo anotace

			\includegraphics[width=0.24\linewidth]{obrazky/seznam}\hfill
			Ukázka pokročilého seznamu prvků. % todo anotace

			\includegraphics[width=0.24\linewidth]{obrazky/vnoreny_seznam}\hfill
			Ukázka vnořeného seznamu uvnitř jiného seznamu. % todo anotace

			\paragraph{Stránkování}

			Při velkém množství položek v jednom seznamu je běžně používaný systém stránkování, kdy se položky rozdělí
			do stránek s omezeným počtem prvků pro jednodušší navigaci po stránce bez nutnosti příliš pohybovat myší.
			Prvek stránkování umožňuje přepínat mezi konkrétními stránkami nebo se přepínat mezi předchozí a následující.
			Pokud je velké množství stránek, prvek nezobrazí všechny stránky, ale zobrazí pouze okolí vybrané stránky
			a první a poslední stránky pro rychlou navigaci.

			\includegraphics[width=0.24\linewidth]{obrazky/strankovani}\hfill
			Ukázka stránkování. % todo anotace

		\subsubsection{Finální návrh}

		% TODO možná konkrétní stránky? figma není úplně aktuální a řekl bych že ten finální design u nění asi úplně to
		% podstatný, myšlenkové postupy už jsou vysvětleny v přechozích kapitolách
		S pomocí výše navržených základních prvků a jejich variacemi bylo navrženo finální \ac{UI} všech stěžejních
		stránek.

		\includegraphics[width=0.24\linewidth]{obrazky/finalni_navrh}\hfill
		Finální grafický návrh aplikačních stránek a prvků. % todo anotace

		Tento postup s návrhem aplikace před samotnou aplikací se ukázal jako velmi efektivní oproti původnímu prototypu
		pomohl nejen snadněji implementovat samotné \ac{GUI}, ale i navrhnout \ac{API} díky zřetelným požadavkům na
		manipulaci s daty.
		\ac{API} tak bylo částečně navrženo právě podle designového návrhu, protože návrh mimo jiné umožnil realizovat
		vizi požadavků na \ac{UI} i funkce aplikace.

	\subsection{Datový model}

	Úkol API server je poskytovat a zpracovávat data ať už textová v pobobě \ac{JSON} dokumentů nebo binární v podobě
	obrázků.
	Tyto funkce pro samotné API bude zprostředkovávat implementovaná business logika starající se, aby všechna data
	byla správně zpracována.
	Je ale stěžejní se na počátku zamyslet a vybrat styl a navrhnout model jakým bude business logika implementována.
	Stěžejní je to z důvodu budoucí čitelnosti kódu a možnostech rozšíření o nové funkce, protože při špatném
	návrhu může být složité aplikace rozšířit a kód se tak může stávat postupně nečitelným.

	Vzhledem k tomu, že byl vybrán jazyk Java, který je primárně objektově orientovaný programovací jazyk, business
	logika bude využívat právě objektově orientovaného programování \ac{OOP}.
	\ac{OOP} je velice populární a většinově používané obecné programovací paradigma využívané spousty programovacími
	jazyky.
	Toto paradigma využívá objekty pro reprezentaci celého systému (aplikace).
	Každý objekt má nějaké své vlastnosti a operace, které může buď využívat vnitřně nebo poskytnou k využití ostatním
	objektům.
	Hlavní myšlenka \ac{OOP} je popisovat reálné objekty (člověk, pes, budova, událost...) těmi virtuálními ve zjednodušené
	formě, kdy popisujeme jen tu část reálného objektu, která nás zajímá.
	Tyto objekty pak v systému mezi sebou komunikují pomocí zpráv a předávají si data, tak aby dosáhli výsledku
	očekávaného od celého systému.
	Každý objekt v systému má svou předlohu zvanou třída definující pro každý objekt jako vlastnosti a operace daný objekt
	může obsahovat a objekt je pak už konkrétní instance obsahující konkrétní data.
	Kromě tříd ještě existují rozhraní a abstraktní třídy, které můžou mezi sebou dědit vlastnosti a operace a z nich
	následně vznikají konkrétní třídy.
	Rozhraní a abstraktní třídy definují obecnou předlohu operací, tzv. kontrakt, který dědící třída musí splňovat, což
	následně umožňuje využít techniku polymorfismus, tj. provádět operace na základě obecného rozhraní bez nutnosti
	vědět konkrétní a třídu a její chování.
	\cite{oop}

	Právě díky \ac{OOP} bylo možné poměrně věrně reprezentovat objekty využívané jednotlivými stránkami v logické,
	přehledné a jasně definované struktuře.
	Model navrhovaného systému původně vycházel z modelu použitém v prototypu, avšak po hlubší analýze požadavků byl
	značně přepracován.
	Nový model byl pro přehlednost interně rozdělen na několik modulů podle jejich zaměření, nicméně všechny moduly
	mezi sebou i tak komunikují, ale nejsou spolu hluboce provázány.
	Momentálně se datový model skládá z modulu pro správu uživatelských účtů, modulu pro správu karet, modulu pro správu
	kupónů pro prémiové plány a podpůrného modulu.

	Modul uživatelských účtů je zodpovědný za vytváření, uchování, úpravu a mazání jednotlivých účtů a jejich dat
	každého jednoho uživatele.
	S tím je spojena i správa oblíbených karet a poznámek, které jsou vždy na konkrétního uživatele a propojují
	moduly účtů a karet.
	Modul dál řeší i prémiový plán, předplatné, expiraci a podobně s tím spojené.
	TODO uml

	\subsection{Přístup k databázi}

	\subsection{DDD}

	TODO
	DDD zmínít až v závěru protože to nepoužívám úplně dobře, takže možná to spojit s datovým modelem

	\subsection{Propojení FE a BE}

	\subsubsection{REST API}

	TODO

	\subsection{Unit testy}

	TODO

	\subsection{Validace odkazů sociálních sítí}

	\subsection{Přihlašování (JWT/SESSION)}

	TODO

	\subsection{GDPR}

	\subsection{Docker a Kubernetes}

	% TODO
%	ostatní věci, konkretní implementace, knihovny v FE a řešení určitých interakcí, stránkování, jak to funguje a co nabízí, notifikace
vycházet z modulů v javě - mail, search, sercurity,storage, validation, app config, testy

% ####################################
\section{Závěr}

% TODO
DDD
digital ocean si občas prostě něco přenastaví pokud to není v souboru o čemž se jentak nedozvíš

% todo
% - změnit vzhled FE
% - vymazat todočka z projektů
% - vymazat WIP todočka (hlavně z HP)
% - smazat emaily
% - smazat hesla
% - možná omezit sítě jen na ty co mají nějaké zvláštnosti
% - zajištění přístupu ke službám třetích stran?
% - smazat statistiky
% - připravit český návod pro rozjetí lokálně všeho komplet, asi i s hosts a tak